%
% API Documentation for API Documentation
% Module tracetool
%
% Generated by epydoc 3.0.1
% [Sat Jan 30 08:22:14 2010]
%

%%%%%%%%%%%%%%%%%%%%%%%%%%%%%%%%%%%%%%%%%%%%%%%%%%%%%%%%%%%%%%%%%%%%%%%%%%%
%%                          Module Description                           %%
%%%%%%%%%%%%%%%%%%%%%%%%%%%%%%%%%%%%%%%%%%%%%%%%%%%%%%%%%%%%%%%%%%%%%%%%%%%

    \index{tracetool \textit{(module)}|(}
\section{Module tracetool}

    \label{tracetool}
tracetool API for python

\textbf{Version:} 12.0



\textbf{Contact:} http://www.codeproject.com/KB/trace/tracetool.aspx



\textbf{Author:} Thierry Parent



\textbf{Copyright:} Copyright (C) 2010 Thierry Parent



\textbf{License:} see License.txt for license information

Sample Use:

\begin{alltt}
\pysrcprompt{{\textgreater}{\textgreater}{\textgreater} }\pysrckeyword{from} tracetool \pysrckeyword{import} ttrace
\pysrcprompt{{\textgreater}{\textgreater}{\textgreater} }ttrace.debug.send (\pysrcstring{"hello world"})\end{alltt}



%%%%%%%%%%%%%%%%%%%%%%%%%%%%%%%%%%%%%%%%%%%%%%%%%%%%%%%%%%%%%%%%%%%%%%%%%%%
%%                               Variables                               %%
%%%%%%%%%%%%%%%%%%%%%%%%%%%%%%%%%%%%%%%%%%%%%%%%%%%%%%%%%%%%%%%%%%%%%%%%%%%

  \subsection{Variables}

    \vspace{-1cm}
\hspace{\varindent}\begin{longtable}{|p{\varnamewidth}|p{\vardescrwidth}|l}
\cline{1-2}
\cline{1-2} \centering \textbf{Name} & \centering \textbf{Description}& \\
\cline{1-2}
\endhead\cline{1-2}\multicolumn{3}{r}{\small\textit{continued on next page}}\\\endfoot\cline{1-2}
\endlastfoot\raggedright \_\-\_\-p\-a\-c\-k\-a\-g\-e\-\_\-\_\- & \raggedright \textbf{Value:} 
{\tt None}&\\
\cline{1-2}
\end{longtable}


%%%%%%%%%%%%%%%%%%%%%%%%%%%%%%%%%%%%%%%%%%%%%%%%%%%%%%%%%%%%%%%%%%%%%%%%%%%
%%                           Class Description                           %%
%%%%%%%%%%%%%%%%%%%%%%%%%%%%%%%%%%%%%%%%%%%%%%%%%%%%%%%%%%%%%%%%%%%%%%%%%%%

    \index{tracetool \textit{(module)}!tracetool.TraceNodeBase \textit{(class)}|(}
\subsection{Class TraceNodeBase}

    \label{tracetool:TraceNodeBase}
\begin{tabular}{cccccc}
% Line for object, linespec=[False]
\multicolumn{2}{r}{\settowidth{\BCL}{object}\multirow{2}{\BCL}{object}}
&&
  \\\cline{3-3}
  &&\multicolumn{1}{c|}{}
&&
  \\
&&\multicolumn{2}{l}{\textbf{tracetool.TraceNodeBase}}
\end{tabular}

\textbf{Known Subclasses:}
tracetool.TraceToSend,
    tracetool.TraceNodeEx

Base class for TraceSend and TraceNodeEx


%%%%%%%%%%%%%%%%%%%%%%%%%%%%%%%%%%%%%%%%%%%%%%%%%%%%%%%%%%%%%%%%%%%%%%%%%%%
%%                                Methods                                %%
%%%%%%%%%%%%%%%%%%%%%%%%%%%%%%%%%%%%%%%%%%%%%%%%%%%%%%%%%%%%%%%%%%%%%%%%%%%

  \subsubsection{Methods}

    \vspace{0.5ex}

\hspace{.8\funcindent}\begin{boxedminipage}{\funcwidth}

    \raggedright \textbf{\_\_init\_\_}(\textit{self})

    \vspace{-1.5ex}

    \rule{\textwidth}{0.5\fboxrule}
\setlength{\parskip}{2ex}
    Internal constructor.Don't create direct instance.See TraceToSend and 
    TraceNodeEx

\setlength{\parskip}{1ex}
      Overrides: object.\_\_init\_\_

    \end{boxedminipage}


\large{\textbf{\textit{Inherited from object}}}

\begin{quote}
\_\_delattr\_\_(), \_\_format\_\_(), \_\_getattribute\_\_(), \_\_hash\_\_(), \_\_new\_\_(), \_\_reduce\_\_(), \_\_reduce\_ex\_\_(), \_\_repr\_\_(), \_\_setattr\_\_(), \_\_sizeof\_\_(), \_\_str\_\_(), \_\_subclasshook\_\_()
\end{quote}

%%%%%%%%%%%%%%%%%%%%%%%%%%%%%%%%%%%%%%%%%%%%%%%%%%%%%%%%%%%%%%%%%%%%%%%%%%%
%%                              Properties                               %%
%%%%%%%%%%%%%%%%%%%%%%%%%%%%%%%%%%%%%%%%%%%%%%%%%%%%%%%%%%%%%%%%%%%%%%%%%%%

  \subsubsection{Properties}

    \vspace{-1cm}
\hspace{\varindent}\begin{longtable}{|p{\varnamewidth}|p{\vardescrwidth}|l}
\cline{1-2}
\cline{1-2} \centering \textbf{Name} & \centering \textbf{Description}& \\
\cline{1-2}
\endhead\cline{1-2}\multicolumn{3}{r}{\small\textit{continued on next page}}\\\endfoot\cline{1-2}
\endlastfoot\multicolumn{2}{|l|}{\textit{Inherited from object}}\\
\multicolumn{2}{|p{\varwidth}|}{\raggedright \_\_class\_\_}\\
\cline{1-2}
\end{longtable}


%%%%%%%%%%%%%%%%%%%%%%%%%%%%%%%%%%%%%%%%%%%%%%%%%%%%%%%%%%%%%%%%%%%%%%%%%%%
%%                          Instance Variables                           %%
%%%%%%%%%%%%%%%%%%%%%%%%%%%%%%%%%%%%%%%%%%%%%%%%%%%%%%%%%%%%%%%%%%%%%%%%%%%

  \subsubsection{Instance Variables}

    \vspace{-1cm}
\hspace{\varindent}\begin{longtable}{|p{\varnamewidth}|p{\vardescrwidth}|l}
\cline{1-2}
\cline{1-2} \centering \textbf{Name} & \centering \textbf{Description}& \\
\cline{1-2}
\endhead\cline{1-2}\multicolumn{3}{r}{\small\textit{continued on next page}}\\\endfoot\cline{1-2}
\endlastfoot\raggedright i\-d\- & The unique ID. Normally it's a GUID/UUID, but can be replaced by 
          something else for inter process traces.&\\
\cline{1-2}
\raggedright e\-n\-a\-b\-l\-e\-d\- & When enabled is false, all traces are disabled. Default is true. 
          All node have a enabled property, that lets you define group of 
          enabled trace. For example set the TTrace.Debug.enabled to false 
          but continue to accept Error and Warning traces&\\
\cline{1-2}
\raggedright w\-i\-n\-T\-r\-a\-c\-e\-I\-d\- & The parent win tree Id&\\
\cline{1-2}
\raggedright t\-a\-g\- & User variable, provided for the convenience of developers&\\
\cline{1-2}
\raggedright i\-c\-o\-n\-I\-n\-d\-e\-x\- & The index of the icon to use. You can then show an icon for 
          Warning traces different for Error traces&\\
\cline{1-2}
\end{longtable}

    \index{tracetool \textit{(module)}!tracetool.TraceNodeBase \textit{(class)}|)}

%%%%%%%%%%%%%%%%%%%%%%%%%%%%%%%%%%%%%%%%%%%%%%%%%%%%%%%%%%%%%%%%%%%%%%%%%%%
%%                           Class Description                           %%
%%%%%%%%%%%%%%%%%%%%%%%%%%%%%%%%%%%%%%%%%%%%%%%%%%%%%%%%%%%%%%%%%%%%%%%%%%%

    \index{tracetool \textit{(module)}!tracetool.NodeContext \textit{(class)}|(}
\subsection{Class NodeContext}

    \label{tracetool:NodeContext}
\begin{tabular}{cccccc}
% Line for object, linespec=[False]
\multicolumn{2}{r}{\settowidth{\BCL}{object}\multirow{2}{\BCL}{object}}
&&
  \\\cline{3-3}
  &&\multicolumn{1}{c|}{}
&&
  \\
&&\multicolumn{2}{l}{\textbf{tracetool.NodeContext}}
\end{tabular}

internal class used to remind indentation node id


%%%%%%%%%%%%%%%%%%%%%%%%%%%%%%%%%%%%%%%%%%%%%%%%%%%%%%%%%%%%%%%%%%%%%%%%%%%
%%                                Methods                                %%
%%%%%%%%%%%%%%%%%%%%%%%%%%%%%%%%%%%%%%%%%%%%%%%%%%%%%%%%%%%%%%%%%%%%%%%%%%%

  \subsubsection{Methods}

    \vspace{0.5ex}

\hspace{.8\funcindent}\begin{boxedminipage}{\funcwidth}

    \raggedright \textbf{\_\_init\_\_}(\textit{self})

\setlength{\parskip}{2ex}
    x.\_\_init\_\_(...) initializes x; see x.\_\_class\_\_.\_\_doc\_\_ for 
    signature

\setlength{\parskip}{1ex}
      Overrides: object.\_\_init\_\_ 	extit{(inherited documentation)}

    \end{boxedminipage}


\large{\textbf{\textit{Inherited from object}}}

\begin{quote}
\_\_delattr\_\_(), \_\_format\_\_(), \_\_getattribute\_\_(), \_\_hash\_\_(), \_\_new\_\_(), \_\_reduce\_\_(), \_\_reduce\_ex\_\_(), \_\_repr\_\_(), \_\_setattr\_\_(), \_\_sizeof\_\_(), \_\_str\_\_(), \_\_subclasshook\_\_()
\end{quote}

%%%%%%%%%%%%%%%%%%%%%%%%%%%%%%%%%%%%%%%%%%%%%%%%%%%%%%%%%%%%%%%%%%%%%%%%%%%
%%                              Properties                               %%
%%%%%%%%%%%%%%%%%%%%%%%%%%%%%%%%%%%%%%%%%%%%%%%%%%%%%%%%%%%%%%%%%%%%%%%%%%%

  \subsubsection{Properties}

    \vspace{-1cm}
\hspace{\varindent}\begin{longtable}{|p{\varnamewidth}|p{\vardescrwidth}|l}
\cline{1-2}
\cline{1-2} \centering \textbf{Name} & \centering \textbf{Description}& \\
\cline{1-2}
\endhead\cline{1-2}\multicolumn{3}{r}{\small\textit{continued on next page}}\\\endfoot\cline{1-2}
\endlastfoot\multicolumn{2}{|l|}{\textit{Inherited from object}}\\
\multicolumn{2}{|p{\varwidth}|}{\raggedright \_\_class\_\_}\\
\cline{1-2}
\end{longtable}

    \index{tracetool \textit{(module)}!tracetool.NodeContext \textit{(class)}|)}

%%%%%%%%%%%%%%%%%%%%%%%%%%%%%%%%%%%%%%%%%%%%%%%%%%%%%%%%%%%%%%%%%%%%%%%%%%%
%%                           Class Description                           %%
%%%%%%%%%%%%%%%%%%%%%%%%%%%%%%%%%%%%%%%%%%%%%%%%%%%%%%%%%%%%%%%%%%%%%%%%%%%

    \index{tracetool \textit{(module)}!tracetool.FontDetail \textit{(class)}|(}
\subsection{Class FontDetail}

    \label{tracetool:FontDetail}
\begin{tabular}{cccccc}
% Line for object, linespec=[False]
\multicolumn{2}{r}{\settowidth{\BCL}{object}\multirow{2}{\BCL}{object}}
&&
  \\\cline{3-3}
  &&\multicolumn{1}{c|}{}
&&
  \\
&&\multicolumn{2}{l}{\textbf{tracetool.FontDetail}}
\end{tabular}

Font detail for traces and members


%%%%%%%%%%%%%%%%%%%%%%%%%%%%%%%%%%%%%%%%%%%%%%%%%%%%%%%%%%%%%%%%%%%%%%%%%%%
%%                                Methods                                %%
%%%%%%%%%%%%%%%%%%%%%%%%%%%%%%%%%%%%%%%%%%%%%%%%%%%%%%%%%%%%%%%%%%%%%%%%%%%

  \subsubsection{Methods}

    \vspace{0.5ex}

\hspace{.8\funcindent}\begin{boxedminipage}{\funcwidth}

    \raggedright \textbf{\_\_init\_\_}(\textit{self})

    \vspace{-1.5ex}

    \rule{\textwidth}{0.5\fboxrule}
\setlength{\parskip}{2ex}
    Construct a default FontDetail object

\setlength{\parskip}{1ex}
      Overrides: object.\_\_init\_\_

    \end{boxedminipage}


\large{\textbf{\textit{Inherited from object}}}

\begin{quote}
\_\_delattr\_\_(), \_\_format\_\_(), \_\_getattribute\_\_(), \_\_hash\_\_(), \_\_new\_\_(), \_\_reduce\_\_(), \_\_reduce\_ex\_\_(), \_\_repr\_\_(), \_\_setattr\_\_(), \_\_sizeof\_\_(), \_\_str\_\_(), \_\_subclasshook\_\_()
\end{quote}

%%%%%%%%%%%%%%%%%%%%%%%%%%%%%%%%%%%%%%%%%%%%%%%%%%%%%%%%%%%%%%%%%%%%%%%%%%%
%%                              Properties                               %%
%%%%%%%%%%%%%%%%%%%%%%%%%%%%%%%%%%%%%%%%%%%%%%%%%%%%%%%%%%%%%%%%%%%%%%%%%%%

  \subsubsection{Properties}

    \vspace{-1cm}
\hspace{\varindent}\begin{longtable}{|p{\varnamewidth}|p{\vardescrwidth}|l}
\cline{1-2}
\cline{1-2} \centering \textbf{Name} & \centering \textbf{Description}& \\
\cline{1-2}
\endhead\cline{1-2}\multicolumn{3}{r}{\small\textit{continued on next page}}\\\endfoot\cline{1-2}
\endlastfoot\multicolumn{2}{|l|}{\textit{Inherited from object}}\\
\multicolumn{2}{|p{\varwidth}|}{\raggedright \_\_class\_\_}\\
\cline{1-2}
\end{longtable}


%%%%%%%%%%%%%%%%%%%%%%%%%%%%%%%%%%%%%%%%%%%%%%%%%%%%%%%%%%%%%%%%%%%%%%%%%%%
%%                          Instance Variables                           %%
%%%%%%%%%%%%%%%%%%%%%%%%%%%%%%%%%%%%%%%%%%%%%%%%%%%%%%%%%%%%%%%%%%%%%%%%%%%

  \subsubsection{Instance Variables}

    \vspace{-1cm}
\hspace{\varindent}\begin{longtable}{|p{\varnamewidth}|p{\vardescrwidth}|l}
\cline{1-2}
\cline{1-2} \centering \textbf{Name} & \centering \textbf{Description}& \\
\cline{1-2}
\endhead\cline{1-2}\multicolumn{3}{r}{\small\textit{continued on next page}}\\\endfoot\cline{1-2}
\endlastfoot\raggedright c\-o\-l\-I\-d\- & column id on where to apply special font&\\
\cline{1-2}
\raggedright b\-o\-l\-d\- & use bold&\\
\cline{1-2}
\raggedright i\-t\-a\-l\-i\-c\- & use italic&\\
\cline{1-2}
\raggedright c\-o\-l\-o\-r\- & color code to use&\\
\cline{1-2}
\raggedright s\-i\-z\-e\- & font size&\\
\cline{1-2}
\raggedright f\-o\-n\-t\-N\-a\-m\-e\- & font name&\\
\cline{1-2}
\end{longtable}

    \index{tracetool \textit{(module)}!tracetool.FontDetail \textit{(class)}|)}

%%%%%%%%%%%%%%%%%%%%%%%%%%%%%%%%%%%%%%%%%%%%%%%%%%%%%%%%%%%%%%%%%%%%%%%%%%%
%%                           Class Description                           %%
%%%%%%%%%%%%%%%%%%%%%%%%%%%%%%%%%%%%%%%%%%%%%%%%%%%%%%%%%%%%%%%%%%%%%%%%%%%

    \index{tracetool \textit{(module)}!tracetool.TMemberNode \textit{(class)}|(}
\subsection{Class TMemberNode}

    \label{tracetool:TMemberNode}
\begin{tabular}{cccccc}
% Line for object, linespec=[False]
\multicolumn{2}{r}{\settowidth{\BCL}{object}\multirow{2}{\BCL}{object}}
&&
  \\\cline{3-3}
  &&\multicolumn{1}{c|}{}
&&
  \\
&&\multicolumn{2}{l}{\textbf{tracetool.TMemberNode}}
\end{tabular}

TMemberNode represent a sub node information in the "info" trace tree. See 
TraceNode class for sample use


%%%%%%%%%%%%%%%%%%%%%%%%%%%%%%%%%%%%%%%%%%%%%%%%%%%%%%%%%%%%%%%%%%%%%%%%%%%
%%                                Methods                                %%
%%%%%%%%%%%%%%%%%%%%%%%%%%%%%%%%%%%%%%%%%%%%%%%%%%%%%%%%%%%%%%%%%%%%%%%%%%%

  \subsubsection{Methods}

    \vspace{0.5ex}

\hspace{.8\funcindent}\begin{boxedminipage}{\funcwidth}

    \raggedright \textbf{\_\_init\_\_}(\textit{self}, \textit{col1}={\tt \texttt{'}\texttt{}\texttt{'}}, \textit{col2}={\tt \texttt{'}\texttt{}\texttt{'}}, \textit{col3}={\tt \texttt{'}\texttt{}\texttt{'}})

\setlength{\parskip}{2ex}
    x.\_\_init\_\_(...) initializes x; see x.\_\_class\_\_.\_\_doc\_\_ for 
    signature

\setlength{\parskip}{1ex}
      Overrides: object.\_\_init\_\_ 	extit{(inherited documentation)}

    \end{boxedminipage}

    \label{tracetool:TMemberNode:add}
    \index{tracetool \textit{(module)}!tracetool.TMemberNode \textit{(class)}!tracetool.TMemberNode.add \textit{(method)}}

    \vspace{0.5ex}

\hspace{.8\funcindent}\begin{boxedminipage}{\funcwidth}

    \raggedright \textbf{add}(\textit{self}, \textit{col1OrMember}, \textit{col2}={\tt \texttt{'}\texttt{}\texttt{'}}, \textit{col3}={\tt \texttt{'}\texttt{}\texttt{'}})

\setlength{\parskip}{2ex}
\setlength{\parskip}{1ex}
    \end{boxedminipage}

    \label{tracetool:TMemberNode:setFontDetail}
    \index{tracetool \textit{(module)}!tracetool.TMemberNode \textit{(class)}!tracetool.TMemberNode.setFontDetail \textit{(method)}}

    \vspace{0.5ex}

\hspace{.8\funcindent}\begin{boxedminipage}{\funcwidth}

    \raggedright \textbf{setFontDetail}(\textit{self}, \textit{colId}, \textit{bold}={\tt False}, \textit{italic}={\tt False}, \textit{color}={\tt -1}, \textit{size}={\tt 0}, \textit{fontName}={\tt \texttt{'}\texttt{}\texttt{'}})

    \vspace{-1.5ex}

    \rule{\textwidth}{0.5\fboxrule}
\setlength{\parskip}{2ex}
    Change font detail for an item in the trace

\setlength{\parskip}{1ex}
      \textbf{Parameters}
      \vspace{-1ex}

      \begin{quote}
        \begin{Ventry}{xxxxxxxx}

          \item[colId]

          Column index : All columns=-1, Col1=0, Col2=1, Col3=2

          \item[bold]

          Change font to bold

          \item[italic]

          Change font to Italic

          \item[color]

          Change Color : -1 for the default color or tuple (R,G,B)

          \item[size]

          Change font size, use zero to keep normal size

          \item[fontName]

          Change font name

        \end{Ventry}

      \end{quote}

      \textbf{Return Value}
    \vspace{-1ex}

      \begin{quote}
      The TMember node

      \end{quote}

    \end{boxedminipage}

    \label{tracetool:TMemberNode:addToStringList}
    \index{tracetool \textit{(module)}!tracetool.TMemberNode \textit{(class)}!tracetool.TMemberNode.addToStringList \textit{(method)}}

    \vspace{0.5ex}

\hspace{.8\funcindent}\begin{boxedminipage}{\funcwidth}

    \raggedright \textbf{addToStringList}(\textit{self}, \textit{commandList})

    \vspace{-1.5ex}

    \rule{\textwidth}{0.5\fboxrule}
\setlength{\parskip}{2ex}
    recursively add members to the node commandList

\setlength{\parskip}{1ex}
      \textbf{Parameters}
      \vspace{-1ex}

      \begin{quote}
        \begin{Ventry}{xxxxxxxxxxx}

          \item[commandList]

          Where to store members

        \end{Ventry}

      \end{quote}

    \end{boxedminipage}


\large{\textbf{\textit{Inherited from object}}}

\begin{quote}
\_\_delattr\_\_(), \_\_format\_\_(), \_\_getattribute\_\_(), \_\_hash\_\_(), \_\_new\_\_(), \_\_reduce\_\_(), \_\_reduce\_ex\_\_(), \_\_repr\_\_(), \_\_setattr\_\_(), \_\_sizeof\_\_(), \_\_str\_\_(), \_\_subclasshook\_\_()
\end{quote}

%%%%%%%%%%%%%%%%%%%%%%%%%%%%%%%%%%%%%%%%%%%%%%%%%%%%%%%%%%%%%%%%%%%%%%%%%%%
%%                              Properties                               %%
%%%%%%%%%%%%%%%%%%%%%%%%%%%%%%%%%%%%%%%%%%%%%%%%%%%%%%%%%%%%%%%%%%%%%%%%%%%

  \subsubsection{Properties}

    \vspace{-1cm}
\hspace{\varindent}\begin{longtable}{|p{\varnamewidth}|p{\vardescrwidth}|l}
\cline{1-2}
\cline{1-2} \centering \textbf{Name} & \centering \textbf{Description}& \\
\cline{1-2}
\endhead\cline{1-2}\multicolumn{3}{r}{\small\textit{continued on next page}}\\\endfoot\cline{1-2}
\endlastfoot\multicolumn{2}{|l|}{\textit{Inherited from object}}\\
\multicolumn{2}{|p{\varwidth}|}{\raggedright \_\_class\_\_}\\
\cline{1-2}
\end{longtable}


%%%%%%%%%%%%%%%%%%%%%%%%%%%%%%%%%%%%%%%%%%%%%%%%%%%%%%%%%%%%%%%%%%%%%%%%%%%
%%                          Instance Variables                           %%
%%%%%%%%%%%%%%%%%%%%%%%%%%%%%%%%%%%%%%%%%%%%%%%%%%%%%%%%%%%%%%%%%%%%%%%%%%%

  \subsubsection{Instance Variables}

    \vspace{-1cm}
\hspace{\varindent}\begin{longtable}{|p{\varnamewidth}|p{\vardescrwidth}|l}
\cline{1-2}
\cline{1-2} \centering \textbf{Name} & \centering \textbf{Description}& \\
\cline{1-2}
\endhead\cline{1-2}\multicolumn{3}{r}{\small\textit{continued on next page}}\\\endfoot\cline{1-2}
\endlastfoot\raggedright c\-o\-l\-1\- & Member column 1 text&\\
\cline{1-2}
\raggedright c\-o\-l\-2\- & Member column 2 text&\\
\cline{1-2}
\raggedright c\-o\-l\-3\- & Member column 3 text&\\
\cline{1-2}
\raggedright m\-e\-m\-b\-e\-r\-s\- & sub members&\\
\cline{1-2}
\raggedright t\-a\-g\- & user defined object (not send to the viewer)&\\
\cline{1-2}
\raggedright v\-i\-e\-w\-e\-r\-K\-i\-n\-d\- & viewer kind.

          \begin{itemize}
          \setlength{\parskip}{0.6ex}
            \item 0 : (CST\_VIEWER\_NONE)  : viewer kind : default viewer, no 
              icon

            \item 1 : (CST\_VIEWER\_DUMP)  : dump viewer

            \item 2 : (CST\_VIEWER\_XML)   : xml viewer

            \item 3 : (CST\_VIEWER\_TABLE) : table viewer

            \item 4 : (CST\_VIEWER\_STACK) : stack

            \item 5 : (CST\_VIEWER\_BITMAP): bitmap viewer

            \item 6 : (CST\_VIEWER\_OBJECT): object structure

            \item 7 : (CST\_VIEWER\_VALUE) : object value

            \item 8 : (CST\_VIEWER\_ENTER) : enter method

            \item 9 : (CST\_VIEWER\_EXIT)  : exit method

            \item 10 : (CST\_VIEWER\_TXT)  : text added to default viewer

          \end{itemize}&\\
\cline{1-2}
\end{longtable}

    \index{tracetool \textit{(module)}!tracetool.TMemberNode \textit{(class)}|)}

%%%%%%%%%%%%%%%%%%%%%%%%%%%%%%%%%%%%%%%%%%%%%%%%%%%%%%%%%%%%%%%%%%%%%%%%%%%
%%                           Class Description                           %%
%%%%%%%%%%%%%%%%%%%%%%%%%%%%%%%%%%%%%%%%%%%%%%%%%%%%%%%%%%%%%%%%%%%%%%%%%%%

    \index{tracetool \textit{(module)}!tracetool.TraceToSend \textit{(class)}|(}
\subsection{Class TraceToSend}

    \label{tracetool:TraceToSend}
\begin{tabular}{cccccccc}
% Line for object, linespec=[False, False]
\multicolumn{2}{r}{\settowidth{\BCL}{object}\multirow{2}{\BCL}{object}}
&&
&&
  \\\cline{3-3}
  &&\multicolumn{1}{c|}{}
&&
&&
  \\
% Line for tracetool.TraceNodeBase, linespec=[False]
\multicolumn{4}{r}{\settowidth{\BCL}{tracetool.TraceNodeBase}\multirow{2}{\BCL}{tracetool.TraceNodeBase}}
&&
  \\\cline{5-5}
  &&&&\multicolumn{1}{c|}{}
&&
  \\
&&&&\multicolumn{2}{l}{\textbf{tracetool.TraceToSend}}
\end{tabular}

\textbf{Known Subclasses:}
tracetool.TraceNode,
    tracetool.WinTrace

Base class for TraceNode and WinTrace. Methods of TraceSend can send new 
traces to the viewer, but cannot be re-send The TTrace.debug object ,for 
example, is a TraceSend instance


%%%%%%%%%%%%%%%%%%%%%%%%%%%%%%%%%%%%%%%%%%%%%%%%%%%%%%%%%%%%%%%%%%%%%%%%%%%
%%                                Methods                                %%
%%%%%%%%%%%%%%%%%%%%%%%%%%%%%%%%%%%%%%%%%%%%%%%%%%%%%%%%%%%%%%%%%%%%%%%%%%%

  \subsubsection{Methods}

    \vspace{0.5ex}

\hspace{.8\funcindent}\begin{boxedminipage}{\funcwidth}

    \raggedright \textbf{\_\_init\_\_}(\textit{self})

\setlength{\parskip}{2ex}
    Internal constructor.Don't create direct instance.See TraceToSend and 
    TraceNodeEx

\setlength{\parskip}{1ex}
      Overrides: object.\_\_init\_\_ 	extit{(inherited documentation)}

    \end{boxedminipage}

    \label{tracetool:TraceToSend:getLastContext}
    \index{tracetool \textit{(module)}!tracetool.TraceToSend \textit{(class)}!tracetool.TraceToSend.getLastContext \textit{(method)}}

    \vspace{0.5ex}

\hspace{.8\funcindent}\begin{boxedminipage}{\funcwidth}

    \raggedright \textbf{getLastContext}(\textit{self})

    \vspace{-1.5ex}

    \rule{\textwidth}{0.5\fboxrule}
\setlength{\parskip}{2ex}
    Get the last context.

\setlength{\parskip}{1ex}
      \textbf{Return Value}
    \vspace{-1ex}

      \begin{quote}
      last context for the thread

      \end{quote}

    \end{boxedminipage}

    \label{tracetool:TraceToSend:getLastContextId}
    \index{tracetool \textit{(module)}!tracetool.TraceToSend \textit{(class)}!tracetool.TraceToSend.getLastContextId \textit{(method)}}

    \vspace{0.5ex}

\hspace{.8\funcindent}\begin{boxedminipage}{\funcwidth}

    \raggedright \textbf{getLastContextId}(\textit{self})

    \vspace{-1.5ex}

    \rule{\textwidth}{0.5\fboxrule}
\setlength{\parskip}{2ex}
    Get the last context ID.

\setlength{\parskip}{1ex}
      \textbf{Return Value}
    \vspace{-1ex}

      \begin{quote}
      last context ID for the thread

      \end{quote}

    \end{boxedminipage}

    \label{tracetool:TraceToSend:pushContext}
    \index{tracetool \textit{(module)}!tracetool.TraceToSend \textit{(class)}!tracetool.TraceToSend.pushContext \textit{(method)}}

    \vspace{0.5ex}

\hspace{.8\funcindent}\begin{boxedminipage}{\funcwidth}

    \raggedright \textbf{pushContext}(\textit{self}, \textit{newContext})

    \vspace{-1.5ex}

    \rule{\textwidth}{0.5\fboxrule}
\setlength{\parskip}{2ex}
    Save the context

\setlength{\parskip}{1ex}
      \textbf{Parameters}
      \vspace{-1ex}

      \begin{quote}
        \begin{Ventry}{xxxxxxxxxx}

          \item[newContext]

          the context to push

        \end{Ventry}

      \end{quote}

    \end{boxedminipage}

    \label{tracetool:TraceToSend:deleteLastContext}
    \index{tracetool \textit{(module)}!tracetool.TraceToSend \textit{(class)}!tracetool.TraceToSend.deleteLastContext \textit{(method)}}

    \vspace{0.5ex}

\hspace{.8\funcindent}\begin{boxedminipage}{\funcwidth}

    \raggedright \textbf{deleteLastContext}(\textit{self})

    \vspace{-1.5ex}

    \rule{\textwidth}{0.5\fboxrule}
\setlength{\parskip}{2ex}
    Delete the last context for the thread

\setlength{\parskip}{1ex}
    \end{boxedminipage}

    \label{tracetool:TraceToSend:getIndentLevel}
    \index{tracetool \textit{(module)}!tracetool.TraceToSend \textit{(class)}!tracetool.TraceToSend.getIndentLevel \textit{(method)}}

    \vspace{0.5ex}

\hspace{.8\funcindent}\begin{boxedminipage}{\funcwidth}

    \raggedright \textbf{getIndentLevel}(\textit{self})

    \vspace{-1.5ex}

    \rule{\textwidth}{0.5\fboxrule}
\setlength{\parskip}{2ex}
    return current indent level. See Indent()

\setlength{\parskip}{1ex}
      \textbf{Return Value}
    \vspace{-1ex}

      \begin{quote}
      current indent level

      \end{quote}

    \end{boxedminipage}

    \label{tracetool:TraceToSend:indent}
    \index{tracetool \textit{(module)}!tracetool.TraceToSend \textit{(class)}!tracetool.TraceToSend.indent \textit{(method)}}

    \vspace{0.5ex}

\hspace{.8\funcindent}\begin{boxedminipage}{\funcwidth}

    \raggedright \textbf{indent}(\textit{self}, \textit{leftMsg}, \textit{rightMsg}={\tt \texttt{'}\texttt{}\texttt{'}}, \textit{backGroundColor}={\tt -1}, \textit{isEnter}={\tt False})

    \vspace{-1.5ex}

    \rule{\textwidth}{0.5\fboxrule}
\setlength{\parskip}{2ex}
    Send a message. further trace to the same node are indented under this 
    one.

\setlength{\parskip}{1ex}
      \textbf{Parameters}
      \vspace{-1ex}

      \begin{quote}
        \begin{Ventry}{xxxxxxxxxxxxxxx}

          \item[rightMsg]

          Right Message to send (optional)

          \item[backGroundColor]

          BackGround Color. -1 for the default color or tuple (R,G,B)

          \item[isEnter]

          if true , a special "enter" icon is added on the node

        \end{Ventry}

      \end{quote}

      \textbf{Return Value}
    \vspace{-1ex}

      \begin{quote}
      The leftMsg : Left message to send

      \end{quote}

    \end{boxedminipage}

    \label{tracetool:TraceToSend:unIndent}
    \index{tracetool \textit{(module)}!tracetool.TraceToSend \textit{(class)}!tracetool.TraceToSend.unIndent \textit{(method)}}

    \vspace{0.5ex}

\hspace{.8\funcindent}\begin{boxedminipage}{\funcwidth}

    \raggedright \textbf{unIndent}(\textit{self}, \textit{leftMsg}={\tt \texttt{'}\texttt{}\texttt{'}}, \textit{rightMsg}={\tt \texttt{'}\texttt{}\texttt{'}}, \textit{backGroundColor}={\tt -1}, \textit{isExit}={\tt False})

    \vspace{-1.5ex}

    \rule{\textwidth}{0.5\fboxrule}
\setlength{\parskip}{2ex}
    Delete indentation to the node added by indent()

\setlength{\parskip}{1ex}
      \textbf{Parameters}
      \vspace{-1ex}

      \begin{quote}
        \begin{Ventry}{xxxxxxxxxxxxxxx}

          \item[leftMsg]

          Left message to send to close indentation (optional)

          \item[rightMsg]

          Right message to send to close indentation (optional)

          \item[backGroundColor]

          background color (optional) -1 for the default color or tuple 
          (R,G,B)

          \item[isExit]

          if true, viewer type 'exit' is used (optional)

        \end{Ventry}

      \end{quote}

    \end{boxedminipage}

    \label{tracetool:TraceToSend:enterMethod}
    \index{tracetool \textit{(module)}!tracetool.TraceToSend \textit{(class)}!tracetool.TraceToSend.enterMethod \textit{(method)}}

    \vspace{0.5ex}

\hspace{.8\funcindent}\begin{boxedminipage}{\funcwidth}

    \raggedright \textbf{enterMethod}(\textit{self}, \textit{leftMsg}={\tt \texttt{'}\texttt{}\texttt{'}}, \textit{rightMsg}={\tt \texttt{'}\texttt{}\texttt{'}}, \textit{backGroundColor}={\tt None})

    \vspace{-1.5ex}

    \rule{\textwidth}{0.5\fboxrule}
\setlength{\parskip}{2ex}
    Indent with "Enter " + left message + right message (optional) + 
    background color (optional)

\setlength{\parskip}{1ex}
      \textbf{Parameters}
      \vspace{-1ex}

      \begin{quote}
        \begin{Ventry}{xxxxxxxxxxxxxxx}

          \item[leftMsg]

          Left message to send

          \item[rightMsg]

          Right message to send

          \item[backGroundColor]

          BackGround Color. -1 for the default color or tuple (R,G,B)

        \end{Ventry}

      \end{quote}

    \end{boxedminipage}

    \label{tracetool:TraceToSend:exitMethod}
    \index{tracetool \textit{(module)}!tracetool.TraceToSend \textit{(class)}!tracetool.TraceToSend.exitMethod \textit{(method)}}

    \vspace{0.5ex}

\hspace{.8\funcindent}\begin{boxedminipage}{\funcwidth}

    \raggedright \textbf{exitMethod}(\textit{self}, \textit{leftMsg}={\tt \texttt{'}\texttt{}\texttt{'}}, \textit{rightMsg}={\tt \texttt{'}\texttt{}\texttt{'}}, \textit{backGroundColor}={\tt None})

    \vspace{-1.5ex}

    \rule{\textwidth}{0.5\fboxrule}
\setlength{\parskip}{2ex}
    UnIndent with "Exit " + left message (optional) + right message 
    (optional) + background color (optional)

\setlength{\parskip}{1ex}
      \textbf{Parameters}
      \vspace{-1ex}

      \begin{quote}
        \begin{Ventry}{xxxxxxxxxxxxxxx}

          \item[leftMsg]

          Left message to send

          \item[rightMsg]

          Right message to send

          \item[backGroundColor]

          BackGround Color. -1 for the default color or tuple (R,G,B)

        \end{Ventry}

      \end{quote}

    \end{boxedminipage}

    \label{tracetool:TraceToSend:prepareNewNode}
    \index{tracetool \textit{(module)}!tracetool.TraceToSend \textit{(class)}!tracetool.TraceToSend.prepareNewNode \textit{(method)}}

    \vspace{0.5ex}

\hspace{.8\funcindent}\begin{boxedminipage}{\funcwidth}

    \raggedright \textbf{prepareNewNode}(\textit{self}, \textit{leftMsg}, \textit{newId}={\tt None})

    \vspace{-1.5ex}

    \rule{\textwidth}{0.5\fboxrule}
\setlength{\parskip}{2ex}
    prepare the minimal command List with leftmsg, trace id, ...

\setlength{\parskip}{1ex}
      \textbf{Parameters}
      \vspace{-1ex}

      \begin{quote}
        \begin{Ventry}{xxxxxxx}

          \item[leftMsg]

          The left message

          \item[newId]

          The trace node ID

        \end{Ventry}

      \end{quote}

      \textbf{Return Value}
    \vspace{-1ex}

      \begin{quote}
      A command list

      \end{quote}

    \end{boxedminipage}

    \label{tracetool:TraceToSend:send}
    \index{tracetool \textit{(module)}!tracetool.TraceToSend \textit{(class)}!tracetool.TraceToSend.send \textit{(method)}}

    \vspace{0.5ex}

\hspace{.8\funcindent}\begin{boxedminipage}{\funcwidth}

    \raggedright \textbf{send}(\textit{self}, \textit{leftMsg}={\tt \texttt{'}\texttt{}\texttt{'}}, \textit{rightMsg}={\tt \texttt{'}\texttt{}\texttt{'}})

    \vspace{-1.5ex}

    \rule{\textwidth}{0.5\fboxrule}
\setlength{\parskip}{2ex}
    The most useful trace function : send just a string

\setlength{\parskip}{1ex}
      \textbf{Parameters}
      \vspace{-1ex}

      \begin{quote}
        \begin{Ventry}{xxxxxxxx}

          \item[leftMsg]

          The message to display

          \item[rightMsg]

          The right message

        \end{Ventry}

      \end{quote}

      \textbf{Return Value}
    \vspace{-1ex}

      \begin{quote}
      A Trace node. Useful to add sub traces

      \end{quote}

    \end{boxedminipage}

    \label{tracetool:TraceToSend:sendValue}
    \index{tracetool \textit{(module)}!tracetool.TraceToSend \textit{(class)}!tracetool.TraceToSend.sendValue \textit{(method)}}

    \vspace{0.5ex}

\hspace{.8\funcindent}\begin{boxedminipage}{\funcwidth}

    \raggedright \textbf{sendValue}(\textit{self}, \textit{leftMsg}, \textit{objToSend}, \textit{sendPrivate}={\tt False}, \textit{maxLevel}={\tt 3}, \textit{title}={\tt None})

    \vspace{-1.5ex}

    \rule{\textwidth}{0.5\fboxrule}
\setlength{\parskip}{2ex}
    Send Private and public values of an object. sendValue is quite 
    different from sendObject : less verbal (no class info) but can show 
    many level.

\setlength{\parskip}{1ex}
      \textbf{Parameters}
      \vspace{-1ex}

      \begin{quote}
        \begin{Ventry}{xxxxxxxxxxx}

          \item[leftMsg]

          The message text

          \item[objToSend]

          the object to examine

          \item[sendPrivate]

          flag to send private field

          \item[maxLevel]

          The number of sub element to display. Default is 3

          \item[title]

          object title

        \end{Ventry}

      \end{quote}

      \textbf{Return Value}
    \vspace{-1ex}

      \begin{quote}
      A trace node

      \end{quote}

    \end{boxedminipage}

    \label{tracetool:TraceToSend:sendObject}
    \index{tracetool \textit{(module)}!tracetool.TraceToSend \textit{(class)}!tracetool.TraceToSend.sendObject \textit{(method)}}

    \vspace{0.5ex}

\hspace{.8\funcindent}\begin{boxedminipage}{\funcwidth}

    \raggedright \textbf{sendObject}(\textit{self}, \textit{leftMsg}, \textit{objToSend}, \textit{bypassFields}={\tt None})

    \vspace{-1.5ex}

    \rule{\textwidth}{0.5\fboxrule}
\setlength{\parskip}{2ex}
    Send a trace and an object (class info, fields, method)

\setlength{\parskip}{1ex}
      \textbf{Parameters}
      \vspace{-1ex}

      \begin{quote}
        \begin{Ventry}{xxxxxxxxx}

          \item[leftMsg]

          The left trace message to send

          \item[objToSend]

          The object to inspect. To specify what to print, see the 
          TraceOption flags

        \end{Ventry}

      \end{quote}

      \textbf{Return Value}
    \vspace{-1ex}

      \begin{quote}
      A trace node

      \end{quote}

    \end{boxedminipage}

    \label{tracetool:TraceToSend:sendStack}
    \index{tracetool \textit{(module)}!tracetool.TraceToSend \textit{(class)}!tracetool.TraceToSend.sendStack \textit{(method)}}

    \vspace{0.5ex}

\hspace{.8\funcindent}\begin{boxedminipage}{\funcwidth}

    \raggedright \textbf{sendStack}(\textit{self}, \textit{leftMsg})

    \vspace{-1.5ex}

    \rule{\textwidth}{0.5\fboxrule}
\setlength{\parskip}{2ex}
    Send the call stack

\setlength{\parskip}{1ex}
      \textbf{Parameters}
      \vspace{-1ex}

      \begin{quote}
        \begin{Ventry}{xxxxxxx}

          \item[leftMsg]

          Trace message

        \end{Ventry}

      \end{quote}

      \textbf{Return Value}
    \vspace{-1ex}

      \begin{quote}
      a Trace node

      \end{quote}

    \end{boxedminipage}

    \label{tracetool:TraceToSend:sendCaller}
    \index{tracetool \textit{(module)}!tracetool.TraceToSend \textit{(class)}!tracetool.TraceToSend.sendCaller \textit{(method)}}

    \vspace{0.5ex}

\hspace{.8\funcindent}\begin{boxedminipage}{\funcwidth}

    \raggedright \textbf{sendCaller}(\textit{self}, \textit{leftMsg})

    \vspace{-1.5ex}

    \rule{\textwidth}{0.5\fboxrule}
\setlength{\parskip}{2ex}
    Send the caller function name

\setlength{\parskip}{1ex}
      \textbf{Parameters}
      \vspace{-1ex}

      \begin{quote}
        \begin{Ventry}{xxxxxxx}

          \item[leftMsg]

          Trace message

        \end{Ventry}

      \end{quote}

      \textbf{Return Value}
    \vspace{-1ex}

      \begin{quote}
      a Trace node

      \end{quote}

    \end{boxedminipage}

    \label{tracetool:TraceToSend:sendDump}
    \index{tracetool \textit{(module)}!tracetool.TraceToSend \textit{(class)}!tracetool.TraceToSend.sendDump \textit{(method)}}

    \vspace{0.5ex}

\hspace{.8\funcindent}\begin{boxedminipage}{\funcwidth}

    \raggedright \textbf{sendDump}(\textit{self}, \textit{leftMsg}, \textit{bytesBuffer}, \textit{shortTitle}={\tt \texttt{'}\texttt{Dump}\texttt{'}}, \textit{count}={\tt 0})

    \vspace{-1.5ex}

    \rule{\textwidth}{0.5\fboxrule}
\setlength{\parskip}{2ex}
    send dump

\setlength{\parskip}{1ex}
      \textbf{Parameters}
      \vspace{-1ex}

      \begin{quote}
        \begin{Ventry}{xxxxxxxxxxx}

          \item[leftMsg]

          Trace message

          \item[shortTitle]

          A short title displayed on top of the dump

          \item[bytesBuffer]

          The byte buffer to dump

          \item[count]

          Number of byte to dump

        \end{Ventry}

      \end{quote}

      \textbf{Return Value}
    \vspace{-1ex}

      \begin{quote}
      a Trace node

      \end{quote}

    \end{boxedminipage}

    \label{tracetool:TraceToSend:sendBackgroundColor}
    \index{tracetool \textit{(module)}!tracetool.TraceToSend \textit{(class)}!tracetool.TraceToSend.sendBackgroundColor \textit{(method)}}

    \vspace{0.5ex}

\hspace{.8\funcindent}\begin{boxedminipage}{\funcwidth}

    \raggedright \textbf{sendBackgroundColor}(\textit{self}, \textit{leftMsg}, \textit{color}={\tt -1}, \textit{colId}={\tt None})

    \vspace{-1.5ex}

    \rule{\textwidth}{0.5\fboxrule}
\setlength{\parskip}{2ex}
    send trace with a specific background color

\setlength{\parskip}{1ex}
      \textbf{Parameters}
      \vspace{-1ex}

      \begin{quote}
        \begin{Ventry}{xxxxxxx}

          \item[leftMsg]

          Trace message

          \item[color]

          background color. -1 for the default color or a tuple (R,G,B)

          \item[colId]

          Column index : All columns=-1, Col1=0, Col2=1, Col3=2

        \end{Ventry}

      \end{quote}

      \textbf{Return Value}
    \vspace{-1ex}

      \begin{quote}
      a Trace node

      \end{quote}

    \end{boxedminipage}

    \label{tracetool:TraceToSend:sendXml}
    \index{tracetool \textit{(module)}!tracetool.TraceToSend \textit{(class)}!tracetool.TraceToSend.sendXml \textit{(method)}}

    \vspace{0.5ex}

\hspace{.8\funcindent}\begin{boxedminipage}{\funcwidth}

    \raggedright \textbf{sendXml}(\textit{self}, \textit{leftMsg}, \textit{xml})

    \vspace{-1.5ex}

    \rule{\textwidth}{0.5\fboxrule}
\setlength{\parskip}{2ex}
    Send xml text

\setlength{\parskip}{1ex}
      \textbf{Parameters}
      \vspace{-1ex}

      \begin{quote}
        \begin{Ventry}{xxxxxxx}

          \item[leftMsg]

          Trace message

          \item[xml]

          xml text to send

        \end{Ventry}

      \end{quote}

      \textbf{Return Value}
    \vspace{-1ex}

      \begin{quote}
      a Trace node

      \end{quote}

    \end{boxedminipage}

    \label{tracetool:TraceToSend:sendTable}
    \index{tracetool \textit{(module)}!tracetool.TraceToSend \textit{(class)}!tracetool.TraceToSend.sendTable \textit{(method)}}

    \vspace{0.5ex}

\hspace{.8\funcindent}\begin{boxedminipage}{\funcwidth}

    \raggedright \textbf{sendTable}(\textit{self}, \textit{leftMsg}, \textit{table})

    \vspace{-1.5ex}

    \rule{\textwidth}{0.5\fboxrule}
\setlength{\parskip}{2ex}
    Add table to node

\setlength{\parskip}{1ex}
      \textbf{Parameters}
      \vspace{-1ex}

      \begin{quote}
        \begin{Ventry}{xxxxxxx}

          \item[leftMsg]

          Trace message

          \item[table]

          TraceTable or Object collection(Array / Collection / Map) to send

        \end{Ventry}

      \end{quote}

      \textbf{Return Value}
    \vspace{-1ex}

      \begin{quote}
      a Trace node

      \end{quote}

    \end{boxedminipage}


\large{\textbf{\textit{Inherited from object}}}

\begin{quote}
\_\_delattr\_\_(), \_\_format\_\_(), \_\_getattribute\_\_(), \_\_hash\_\_(), \_\_new\_\_(), \_\_reduce\_\_(), \_\_reduce\_ex\_\_(), \_\_repr\_\_(), \_\_setattr\_\_(), \_\_sizeof\_\_(), \_\_str\_\_(), \_\_subclasshook\_\_()
\end{quote}

%%%%%%%%%%%%%%%%%%%%%%%%%%%%%%%%%%%%%%%%%%%%%%%%%%%%%%%%%%%%%%%%%%%%%%%%%%%
%%                              Properties                               %%
%%%%%%%%%%%%%%%%%%%%%%%%%%%%%%%%%%%%%%%%%%%%%%%%%%%%%%%%%%%%%%%%%%%%%%%%%%%

  \subsubsection{Properties}

    \vspace{-1cm}
\hspace{\varindent}\begin{longtable}{|p{\varnamewidth}|p{\vardescrwidth}|l}
\cline{1-2}
\cline{1-2} \centering \textbf{Name} & \centering \textbf{Description}& \\
\cline{1-2}
\endhead\cline{1-2}\multicolumn{3}{r}{\small\textit{continued on next page}}\\\endfoot\cline{1-2}
\endlastfoot\multicolumn{2}{|l|}{\textit{Inherited from object}}\\
\multicolumn{2}{|p{\varwidth}|}{\raggedright \_\_class\_\_}\\
\cline{1-2}
\end{longtable}


%%%%%%%%%%%%%%%%%%%%%%%%%%%%%%%%%%%%%%%%%%%%%%%%%%%%%%%%%%%%%%%%%%%%%%%%%%%
%%                          Instance Variables                           %%
%%%%%%%%%%%%%%%%%%%%%%%%%%%%%%%%%%%%%%%%%%%%%%%%%%%%%%%%%%%%%%%%%%%%%%%%%%%

  \subsubsection{Instance Variables}

    \vspace{-1cm}
\hspace{\varindent}\begin{longtable}{|p{\varnamewidth}|p{\vardescrwidth}|l}
\cline{1-2}
\cline{1-2} \centering \textbf{Name} & \centering \textbf{Description}& \\
\cline{1-2}
\endhead\cline{1-2}\multicolumn{3}{r}{\small\textit{continued on next page}}\\\endfoot\cline{1-2}
\endlastfoot\multicolumn{2}{|l|}{\textit{Inherited from tracetool.TraceNodeBase \textit{(Section \ref{tracetool:TraceNodeBase})}}}\\
\multicolumn{2}{|p{\varwidth}|}{\raggedright enabled, iconIndex, id, tag, winTraceId}\\
\cline{1-2}
\end{longtable}

    \index{tracetool \textit{(module)}!tracetool.TraceToSend \textit{(class)}|)}

%%%%%%%%%%%%%%%%%%%%%%%%%%%%%%%%%%%%%%%%%%%%%%%%%%%%%%%%%%%%%%%%%%%%%%%%%%%
%%                           Class Description                           %%
%%%%%%%%%%%%%%%%%%%%%%%%%%%%%%%%%%%%%%%%%%%%%%%%%%%%%%%%%%%%%%%%%%%%%%%%%%%

    \index{tracetool \textit{(module)}!tracetool.TraceNode \textit{(class)}|(}
\subsection{Class TraceNode}

    \label{tracetool:TraceNode}
\begin{tabular}{cccccccccc}
% Line for object, linespec=[False, False, False]
\multicolumn{2}{r}{\settowidth{\BCL}{object}\multirow{2}{\BCL}{object}}
&&
&&
&&
  \\\cline{3-3}
  &&\multicolumn{1}{c|}{}
&&
&&
&&
  \\
% Line for tracetool.TraceNodeBase, linespec=[False, False]
\multicolumn{4}{r}{\settowidth{\BCL}{tracetool.TraceNodeBase}\multirow{2}{\BCL}{tracetool.TraceNodeBase}}
&&
&&
  \\\cline{5-5}
  &&&&\multicolumn{1}{c|}{}
&&
&&
  \\
% Line for tracetool.TraceToSend, linespec=[False]
\multicolumn{6}{r}{\settowidth{\BCL}{tracetool.TraceToSend}\multirow{2}{\BCL}{tracetool.TraceToSend}}
&&
  \\\cline{7-7}
  &&&&&&\multicolumn{1}{c|}{}
&&
  \\
&&&&&&\multicolumn{2}{l}{\textbf{tracetool.TraceNode}}
\end{tabular}


%%%%%%%%%%%%%%%%%%%%%%%%%%%%%%%%%%%%%%%%%%%%%%%%%%%%%%%%%%%%%%%%%%%%%%%%%%%
%%                                Methods                                %%
%%%%%%%%%%%%%%%%%%%%%%%%%%%%%%%%%%%%%%%%%%%%%%%%%%%%%%%%%%%%%%%%%%%%%%%%%%%

  \subsubsection{Methods}

    \vspace{0.5ex}

\hspace{.8\funcindent}\begin{boxedminipage}{\funcwidth}

    \raggedright \textbf{\_\_init\_\_}(\textit{self}, \textit{source}={\tt None}, \textit{generateUniqueId}={\tt False})

\setlength{\parskip}{2ex}
    Internal constructor.Don't create direct instance.See TraceToSend and 
    TraceNodeEx

\setlength{\parskip}{1ex}
      Overrides: object.\_\_init\_\_ 	extit{(inherited documentation)}

    \end{boxedminipage}

    \label{tracetool:TraceNode:resend}
    \index{tracetool \textit{(module)}!tracetool.TraceNode \textit{(class)}!tracetool.TraceNode.resend \textit{(method)}}

    \vspace{0.5ex}

\hspace{.8\funcindent}\begin{boxedminipage}{\funcwidth}

    \raggedright \textbf{resend}(\textit{self}, \textit{newLeftMsg}={\tt None}, \textit{newRightMsg}={\tt None})

    \vspace{-1.5ex}

    \rule{\textwidth}{0.5\fboxrule}
\setlength{\parskip}{2ex}
    Override a previous send message (both column)

\setlength{\parskip}{1ex}
      \textbf{Parameters}
      \vspace{-1ex}

      \begin{quote}
        \begin{Ventry}{xxxxxxxxxxx}

          \item[newLeftMsg]

          The new Left message

          \item[newRightMsg]

          The new Right message

        \end{Ventry}

      \end{quote}

      \textbf{Return Value}
    \vspace{-1ex}

      \begin{quote}
      The trace node

      \end{quote}

    \end{boxedminipage}

    \label{tracetool:TraceNode:resendIconIndex}
    \index{tracetool \textit{(module)}!tracetool.TraceNode \textit{(class)}!tracetool.TraceNode.resendIconIndex \textit{(method)}}

    \vspace{0.5ex}

\hspace{.8\funcindent}\begin{boxedminipage}{\funcwidth}

    \raggedright \textbf{resendIconIndex}(\textit{self}, \textit{index})

    \vspace{-1.5ex}

    \rule{\textwidth}{0.5\fboxrule}
\setlength{\parskip}{2ex}
    Change the Icon index

\setlength{\parskip}{1ex}
      \textbf{Parameters}
      \vspace{-1ex}

      \begin{quote}
        \begin{Ventry}{xxxxx}

          \item[index]

          Index of the icon to use

        \end{Ventry}

      \end{quote}

      \textbf{Return Value}
    \vspace{-1ex}

      \begin{quote}
      The trace node

      \end{quote}

    \end{boxedminipage}

    \label{tracetool:TraceNode:setBackgroundColor}
    \index{tracetool \textit{(module)}!tracetool.TraceNode \textit{(class)}!tracetool.TraceNode.setBackgroundColor \textit{(method)}}

    \vspace{0.5ex}

\hspace{.8\funcindent}\begin{boxedminipage}{\funcwidth}

    \raggedright \textbf{setBackgroundColor}(\textit{self}, \textit{color}={\tt -1}, \textit{colId}={\tt -1})

    \vspace{-1.5ex}

    \rule{\textwidth}{0.5\fboxrule}
\setlength{\parskip}{2ex}
    Change Background Color (whole line) of a node

\setlength{\parskip}{1ex}
      \textbf{Parameters}
      \vspace{-1ex}

      \begin{quote}
        \begin{Ventry}{xxxxx}

          \item[color]

          new background color of the node. -1 for the default color or 
          tuple (R,G,B)

          \item[colId]

          Column index : All columns=-1, Icon=0, Time=1, thread=2, left 
          msg=3, right msg =4 or user defined column

        \end{Ventry}

      \end{quote}

      \textbf{Return Value}
    \vspace{-1ex}

      \begin{quote}
      The trace node

      \end{quote}

    \end{boxedminipage}

    \label{tracetool:TraceNode:setFontDetail}
    \index{tracetool \textit{(module)}!tracetool.TraceNode \textit{(class)}!tracetool.TraceNode.setFontDetail \textit{(method)}}

    \vspace{0.5ex}

\hspace{.8\funcindent}\begin{boxedminipage}{\funcwidth}

    \raggedright \textbf{setFontDetail}(\textit{self}, \textit{colId}={\tt -1}, \textit{bold}={\tt False}, \textit{italic}={\tt False}, \textit{color}={\tt None}, \textit{size}={\tt 0}, \textit{fontName}={\tt \texttt{'}\texttt{}\texttt{'}})

    \vspace{-1.5ex}

    \rule{\textwidth}{0.5\fboxrule}
\setlength{\parskip}{2ex}
    Change font detail for an item in the trace

\setlength{\parskip}{1ex}
      \textbf{Parameters}
      \vspace{-1ex}

      \begin{quote}
        \begin{Ventry}{xxxxxxxx}

          \item[colId]

          Column index : All columns=-1, Icon=0, Time=1, thread=2, left 
          msg=3, right msg =4 or user defined column

          \item[bold]

          Change font to bold

          \item[italic]

          Change font to Italic

          \item[color]

          Change Color.-1 for the default color or a tuple (R,G,B)

          \item[size]

          Change font size, use zero to keep normal size

          \item[fontName]

          Change font name

        \end{Ventry}

      \end{quote}

      \textbf{Return Value}
    \vspace{-1ex}

      \begin{quote}
      The trace node

      \end{quote}

    \end{boxedminipage}

    \label{tracetool:TraceNode:append}
    \index{tracetool \textit{(module)}!tracetool.TraceNode \textit{(class)}!tracetool.TraceNode.append \textit{(method)}}

    \vspace{0.5ex}

\hspace{.8\funcindent}\begin{boxedminipage}{\funcwidth}

    \raggedright \textbf{append}(\textit{self}, \textit{newLeftMsg}={\tt None}, \textit{newRightMsg}={\tt None})

    \vspace{-1.5ex}

    \rule{\textwidth}{0.5\fboxrule}
\setlength{\parskip}{2ex}
    Append text to a previous send message (both column)

\setlength{\parskip}{1ex}
      \textbf{Parameters}
      \vspace{-1ex}

      \begin{quote}
        \begin{Ventry}{xxxxxxxxxxx}

          \item[newLeftMsg]

          The new Left message to append

          \item[newRightMsg]

          The new Right message to append

        \end{Ventry}

      \end{quote}

      \textbf{Return Value}
    \vspace{-1ex}

      \begin{quote}
      The trace node

      \end{quote}

    \end{boxedminipage}

    \label{tracetool:TraceNode:setSelected}
    \index{tracetool \textit{(module)}!tracetool.TraceNode \textit{(class)}!tracetool.TraceNode.setSelected \textit{(method)}}

    \vspace{0.5ex}

\hspace{.8\funcindent}\begin{boxedminipage}{\funcwidth}

    \raggedright \textbf{setSelected}(\textit{self})

    \vspace{-1.5ex}

    \rule{\textwidth}{0.5\fboxrule}
\setlength{\parskip}{2ex}
    Set a node as selected in the viewer. Don't confuse with show() that 
    force a node to be displayed

\setlength{\parskip}{1ex}
      \textbf{Return Value}
    \vspace{-1ex}

      \begin{quote}
      The trace node

      \end{quote}

    \end{boxedminipage}

    \label{tracetool:TraceNode:show}
    \index{tracetool \textit{(module)}!tracetool.TraceNode \textit{(class)}!tracetool.TraceNode.show \textit{(method)}}

    \vspace{0.5ex}

\hspace{.8\funcindent}\begin{boxedminipage}{\funcwidth}

    \raggedright \textbf{show}(\textit{self})

    \vspace{-1.5ex}

    \rule{\textwidth}{0.5\fboxrule}
\setlength{\parskip}{2ex}
    Force a node to be displayed. don't confuse with setSelected()

\setlength{\parskip}{1ex}
      \textbf{Return Value}
    \vspace{-1ex}

      \begin{quote}
      The trace node

      \end{quote}

    \end{boxedminipage}

    \label{tracetool:TraceNode:delete}
    \index{tracetool \textit{(module)}!tracetool.TraceNode \textit{(class)}!tracetool.TraceNode.delete \textit{(method)}}

    \vspace{0.5ex}

\hspace{.8\funcindent}\begin{boxedminipage}{\funcwidth}

    \raggedright \textbf{delete}(\textit{self})

    \vspace{-1.5ex}

    \rule{\textwidth}{0.5\fboxrule}
\setlength{\parskip}{2ex}
    Delete the node

\setlength{\parskip}{1ex}
      \textbf{Return Value}
    \vspace{-1ex}

      \begin{quote}
      The trace node

      \end{quote}

    \end{boxedminipage}

    \label{tracetool:TraceNode:deleteChildren}
    \index{tracetool \textit{(module)}!tracetool.TraceNode \textit{(class)}!tracetool.TraceNode.deleteChildren \textit{(method)}}

    \vspace{0.5ex}

\hspace{.8\funcindent}\begin{boxedminipage}{\funcwidth}

    \raggedright \textbf{deleteChildren}(\textit{self})

    \vspace{-1.5ex}

    \rule{\textwidth}{0.5\fboxrule}
\setlength{\parskip}{2ex}
    Delete children node

\setlength{\parskip}{1ex}
      \textbf{Return Value}
    \vspace{-1ex}

      \begin{quote}
      The trace node

      \end{quote}

    \end{boxedminipage}

    \label{tracetool:TraceNode:setBookmark}
    \index{tracetool \textit{(module)}!tracetool.TraceNode \textit{(class)}!tracetool.TraceNode.setBookmark \textit{(method)}}

    \vspace{0.5ex}

\hspace{.8\funcindent}\begin{boxedminipage}{\funcwidth}

    \raggedright \textbf{setBookmark}(\textit{self}, \textit{bookmarked})

    \vspace{-1.5ex}

    \rule{\textwidth}{0.5\fboxrule}
\setlength{\parskip}{2ex}
    Set or reset the bookmark for the node

\setlength{\parskip}{1ex}
      \textbf{Parameters}
      \vspace{-1ex}

      \begin{quote}
        \begin{Ventry}{xxxxxxxxxx}

          \item[bookmarked]

          true/false

        \end{Ventry}

      \end{quote}

      \textbf{Return Value}
    \vspace{-1ex}

      \begin{quote}
      The trace node

      \end{quote}

    \end{boxedminipage}

    \label{tracetool:TraceNode:setVisible}
    \index{tracetool \textit{(module)}!tracetool.TraceNode \textit{(class)}!tracetool.TraceNode.setVisible \textit{(method)}}

    \vspace{0.5ex}

\hspace{.8\funcindent}\begin{boxedminipage}{\funcwidth}

    \raggedright \textbf{setVisible}(\textit{self}, \textit{visible})

    \vspace{-1.5ex}

    \rule{\textwidth}{0.5\fboxrule}
\setlength{\parskip}{2ex}
    set a node visible or invisible

\setlength{\parskip}{1ex}
      \textbf{Parameters}
      \vspace{-1ex}

      \begin{quote}
        \begin{Ventry}{xxxxxxx}

          \item[visible]

          true/false

        \end{Ventry}

      \end{quote}

      \textbf{Return Value}
    \vspace{-1ex}

      \begin{quote}
      The trace node

      \end{quote}

    \end{boxedminipage}

    \label{tracetool:TraceNode:gotoNextSibling}
    \index{tracetool \textit{(module)}!tracetool.TraceNode \textit{(class)}!tracetool.TraceNode.gotoNextSibling \textit{(method)}}

    \vspace{0.5ex}

\hspace{.8\funcindent}\begin{boxedminipage}{\funcwidth}

    \raggedright \textbf{gotoNextSibling}(\textit{self})

    \vspace{-1.5ex}

    \rule{\textwidth}{0.5\fboxrule}
\setlength{\parskip}{2ex}
    Set focus to next sibling

\setlength{\parskip}{1ex}
      \textbf{Return Value}
    \vspace{-1ex}

      \begin{quote}
      The trace node

      \end{quote}

    \end{boxedminipage}

    \label{tracetool:TraceNode:gotoPrevSibling}
    \index{tracetool \textit{(module)}!tracetool.TraceNode \textit{(class)}!tracetool.TraceNode.gotoPrevSibling \textit{(method)}}

    \vspace{0.5ex}

\hspace{.8\funcindent}\begin{boxedminipage}{\funcwidth}

    \raggedright \textbf{gotoPrevSibling}(\textit{self})

    \vspace{-1.5ex}

    \rule{\textwidth}{0.5\fboxrule}
\setlength{\parskip}{2ex}
    Set focus to previous sibling

\setlength{\parskip}{1ex}
      \textbf{Return Value}
    \vspace{-1ex}

      \begin{quote}
      The trace node

      \end{quote}

    \end{boxedminipage}

    \label{tracetool:TraceNode:gotoFirstChild}
    \index{tracetool \textit{(module)}!tracetool.TraceNode \textit{(class)}!tracetool.TraceNode.gotoFirstChild \textit{(method)}}

    \vspace{0.5ex}

\hspace{.8\funcindent}\begin{boxedminipage}{\funcwidth}

    \raggedright \textbf{gotoFirstChild}(\textit{self})

    \vspace{-1.5ex}

    \rule{\textwidth}{0.5\fboxrule}
\setlength{\parskip}{2ex}
    Set focus to first child

\setlength{\parskip}{1ex}
      \textbf{Return Value}
    \vspace{-1ex}

      \begin{quote}
      The trace node

      \end{quote}

    \end{boxedminipage}

    \label{tracetool:TraceNode:gotoLastChild}
    \index{tracetool \textit{(module)}!tracetool.TraceNode \textit{(class)}!tracetool.TraceNode.gotoLastChild \textit{(method)}}

    \vspace{0.5ex}

\hspace{.8\funcindent}\begin{boxedminipage}{\funcwidth}

    \raggedright \textbf{gotoLastChild}(\textit{self})

    \vspace{-1.5ex}

    \rule{\textwidth}{0.5\fboxrule}
\setlength{\parskip}{2ex}
    Set focus to last child

\setlength{\parskip}{1ex}
      \textbf{Return Value}
    \vspace{-1ex}

      \begin{quote}
      The trace node

      \end{quote}

    \end{boxedminipage}


\large{\textbf{\textit{Inherited from tracetool.TraceToSend\textit{(Section \ref{tracetool:TraceToSend})}}}}

\begin{quote}
deleteLastContext(), enterMethod(), exitMethod(), getIndentLevel(), getLastContext(), getLastContextId(), indent(), prepareNewNode(), pushContext(), send(), sendBackgroundColor(), sendCaller(), sendDump(), sendObject(), sendStack(), sendTable(), sendValue(), sendXml(), unIndent()
\end{quote}

\large{\textbf{\textit{Inherited from object}}}

\begin{quote}
\_\_delattr\_\_(), \_\_format\_\_(), \_\_getattribute\_\_(), \_\_hash\_\_(), \_\_new\_\_(), \_\_reduce\_\_(), \_\_reduce\_ex\_\_(), \_\_repr\_\_(), \_\_setattr\_\_(), \_\_sizeof\_\_(), \_\_str\_\_(), \_\_subclasshook\_\_()
\end{quote}

%%%%%%%%%%%%%%%%%%%%%%%%%%%%%%%%%%%%%%%%%%%%%%%%%%%%%%%%%%%%%%%%%%%%%%%%%%%
%%                              Properties                               %%
%%%%%%%%%%%%%%%%%%%%%%%%%%%%%%%%%%%%%%%%%%%%%%%%%%%%%%%%%%%%%%%%%%%%%%%%%%%

  \subsubsection{Properties}

    \vspace{-1cm}
\hspace{\varindent}\begin{longtable}{|p{\varnamewidth}|p{\vardescrwidth}|l}
\cline{1-2}
\cline{1-2} \centering \textbf{Name} & \centering \textbf{Description}& \\
\cline{1-2}
\endhead\cline{1-2}\multicolumn{3}{r}{\small\textit{continued on next page}}\\\endfoot\cline{1-2}
\endlastfoot\multicolumn{2}{|l|}{\textit{Inherited from object}}\\
\multicolumn{2}{|p{\varwidth}|}{\raggedright \_\_class\_\_}\\
\cline{1-2}
\end{longtable}


%%%%%%%%%%%%%%%%%%%%%%%%%%%%%%%%%%%%%%%%%%%%%%%%%%%%%%%%%%%%%%%%%%%%%%%%%%%
%%                          Instance Variables                           %%
%%%%%%%%%%%%%%%%%%%%%%%%%%%%%%%%%%%%%%%%%%%%%%%%%%%%%%%%%%%%%%%%%%%%%%%%%%%

  \subsubsection{Instance Variables}

    \vspace{-1cm}
\hspace{\varindent}\begin{longtable}{|p{\varnamewidth}|p{\vardescrwidth}|l}
\cline{1-2}
\cline{1-2} \centering \textbf{Name} & \centering \textbf{Description}& \\
\cline{1-2}
\endhead\cline{1-2}\multicolumn{3}{r}{\small\textit{continued on next page}}\\\endfoot\cline{1-2}
\endlastfoot\multicolumn{2}{|l|}{\textit{Inherited from tracetool.TraceNodeBase \textit{(Section \ref{tracetool:TraceNodeBase})}}}\\
\multicolumn{2}{|p{\varwidth}|}{\raggedright enabled, iconIndex, id, tag, winTraceId}\\
\cline{1-2}
\end{longtable}

    \index{tracetool \textit{(module)}!tracetool.TraceNode \textit{(class)}|)}

%%%%%%%%%%%%%%%%%%%%%%%%%%%%%%%%%%%%%%%%%%%%%%%%%%%%%%%%%%%%%%%%%%%%%%%%%%%
%%                           Class Description                           %%
%%%%%%%%%%%%%%%%%%%%%%%%%%%%%%%%%%%%%%%%%%%%%%%%%%%%%%%%%%%%%%%%%%%%%%%%%%%

    \index{tracetool \textit{(module)}!tracetool.TraceNodeEx \textit{(class)}|(}
\subsection{Class TraceNodeEx}

    \label{tracetool:TraceNodeEx}
\begin{tabular}{cccccccc}
% Line for object, linespec=[False, False]
\multicolumn{2}{r}{\settowidth{\BCL}{object}\multirow{2}{\BCL}{object}}
&&
&&
  \\\cline{3-3}
  &&\multicolumn{1}{c|}{}
&&
&&
  \\
% Line for tracetool.TraceNodeBase, linespec=[False]
\multicolumn{4}{r}{\settowidth{\BCL}{tracetool.TraceNodeBase}\multirow{2}{\BCL}{tracetool.TraceNodeBase}}
&&
  \\\cline{5-5}
  &&&&\multicolumn{1}{c|}{}
&&
  \\
&&&&\multicolumn{2}{l}{\textbf{tracetool.TraceNodeEx}}
\end{tabular}


%%%%%%%%%%%%%%%%%%%%%%%%%%%%%%%%%%%%%%%%%%%%%%%%%%%%%%%%%%%%%%%%%%%%%%%%%%%
%%                                Methods                                %%
%%%%%%%%%%%%%%%%%%%%%%%%%%%%%%%%%%%%%%%%%%%%%%%%%%%%%%%%%%%%%%%%%%%%%%%%%%%

  \subsubsection{Methods}

    \vspace{0.5ex}

\hspace{.8\funcindent}\begin{boxedminipage}{\funcwidth}

    \raggedright \textbf{\_\_init\_\_}(\textit{self}, \textit{source}={\tt None}, \textit{generateUniqueId}={\tt False})

\setlength{\parskip}{2ex}
    Internal constructor.Don't create direct instance.See TraceToSend and 
    TraceNodeEx

\setlength{\parskip}{1ex}
      Overrides: object.\_\_init\_\_ 	extit{(inherited documentation)}

    \end{boxedminipage}

    \label{tracetool:TraceNodeEx:addObject}
    \index{tracetool \textit{(module)}!tracetool.TraceNodeEx \textit{(class)}!tracetool.TraceNodeEx.addObject \textit{(method)}}

    \vspace{0.5ex}

\hspace{.8\funcindent}\begin{boxedminipage}{\funcwidth}

    \raggedright \textbf{addObject}(\textit{self}, \textit{objToSend}, \textit{bypassFields}={\tt None})

    \vspace{-1.5ex}

    \rule{\textwidth}{0.5\fboxrule}
\setlength{\parskip}{2ex}
    Call addObject to fill the "member" tree with the object value. To 
    specify what to print, see the TraceOption flags

\setlength{\parskip}{1ex}
      \textbf{Parameters}
      \vspace{-1ex}

      \begin{quote}
        \begin{Ventry}{xxxxxxxxxxxx}

          \item[objToSend]

          the object to send.

          \item[bypassFields]

          Optional array of fieldName to don't print

        \end{Ventry}

      \end{quote}

    \end{boxedminipage}

    \label{tracetool:TraceNodeEx:addValue}
    \index{tracetool \textit{(module)}!tracetool.TraceNodeEx \textit{(class)}!tracetool.TraceNodeEx.addValue \textit{(method)}}

    \vspace{0.5ex}

\hspace{.8\funcindent}\begin{boxedminipage}{\funcwidth}

    \raggedright \textbf{addValue}(\textit{self}, \textit{objToSend}, \textit{sendPrivate}={\tt False}, \textit{maxLevel}={\tt 3}, \textit{title}={\tt None}, \textit{alreadyParsedObject}={\tt None})

    \vspace{-1.5ex}

    \rule{\textwidth}{0.5\fboxrule}
\setlength{\parskip}{2ex}
    Send Private and public values of an object. addValue is quite 
    different from addObject : less verbal (no class info) but can show 
    many level.

\setlength{\parskip}{1ex}
      \textbf{Parameters}
      \vspace{-1ex}

      \begin{quote}
        \begin{Ventry}{xxxxxxxxxxxxxxxxxxx}

          \item[objToSend]

          The object to examine

          \item[sendPrivate]

          flag to send private field

          \item[maxLevel]

          The number of sub element to display. Default is 3

          \item[title]

          Object title

          \item[alreadyParsedObject]

          List of already parsed objects

        \end{Ventry}

      \end{quote}

    \end{boxedminipage}

    \label{tracetool:TraceNodeEx:addTable}
    \index{tracetool \textit{(module)}!tracetool.TraceNodeEx \textit{(class)}!tracetool.TraceNodeEx.addTable \textit{(method)}}

    \vspace{0.5ex}

\hspace{.8\funcindent}\begin{boxedminipage}{\funcwidth}

    \raggedright \textbf{addTable}(\textit{self}, \textit{table})

    \vspace{-1.5ex}

    \rule{\textwidth}{0.5\fboxrule}
\setlength{\parskip}{2ex}
    Add table to node

\setlength{\parskip}{1ex}
      \textbf{Parameters}
      \vspace{-1ex}

      \begin{quote}
        \begin{Ventry}{xxxxx}

          \item[table]

          Object collection(Array / Collection / Map) to send

        \end{Ventry}

      \end{quote}

    \end{boxedminipage}

    \label{tracetool:TraceNodeEx:addDump}
    \index{tracetool \textit{(module)}!tracetool.TraceNodeEx \textit{(class)}!tracetool.TraceNodeEx.addDump \textit{(method)}}

    \vspace{0.5ex}

\hspace{.8\funcindent}\begin{boxedminipage}{\funcwidth}

    \raggedright \textbf{addDump}(\textit{self}, \textit{bytesBuffer}, \textit{shortTitle}={\tt \texttt{'}\texttt{Dump}\texttt{'}}, \textit{index}={\tt 0}, \textit{count}={\tt 0})

    \vspace{-1.5ex}

    \rule{\textwidth}{0.5\fboxrule}
\setlength{\parskip}{2ex}
    Display dump

\setlength{\parskip}{1ex}
      \textbf{Parameters}
      \vspace{-1ex}

      \begin{quote}
        \begin{Ventry}{xxxxxxxxxxx}

          \item[shortTitle]

          A short title displayed on top of the dump

          \item[bytesBuffer]

          The buffer to dump : string (single byte char or unicode) , 
          bytearray or any array like object that has char or int elements

          \item[index]

          start index (default is first)

          \item[count]

          Number of byte to dump (default is all)

        \end{Ventry}

      \end{quote}

    \end{boxedminipage}

    \label{tracetool:TraceNodeEx:addStackTrace}
    \index{tracetool \textit{(module)}!tracetool.TraceNodeEx \textit{(class)}!tracetool.TraceNodeEx.addStackTrace \textit{(method)}}

    \vspace{0.5ex}

\hspace{.8\funcindent}\begin{boxedminipage}{\funcwidth}

    \raggedright \textbf{addStackTrace}(\textit{self})

    \vspace{-1.5ex}

    \rule{\textwidth}{0.5\fboxrule}
\setlength{\parskip}{2ex}
    add stack trace

\setlength{\parskip}{1ex}
    \end{boxedminipage}

    \label{tracetool:TraceNodeEx:addCaller}
    \index{tracetool \textit{(module)}!tracetool.TraceNodeEx \textit{(class)}!tracetool.TraceNodeEx.addCaller \textit{(method)}}

    \vspace{0.5ex}

\hspace{.8\funcindent}\begin{boxedminipage}{\funcwidth}

    \raggedright \textbf{addCaller}(\textit{self})

    \vspace{-1.5ex}

    \rule{\textwidth}{0.5\fboxrule}
\setlength{\parskip}{2ex}
    show caller information. level 0 correspond at the first line of 
    StackTrace(0) It's like the call stack, but display only 1 line

\setlength{\parskip}{1ex}
    \end{boxedminipage}

    \label{tracetool:TraceNodeEx:addXML}
    \index{tracetool \textit{(module)}!tracetool.TraceNodeEx \textit{(class)}!tracetool.TraceNodeEx.addXML \textit{(method)}}

    \vspace{0.5ex}

\hspace{.8\funcindent}\begin{boxedminipage}{\funcwidth}

    \raggedright \textbf{addXML}(\textit{self}, \textit{xml})

    \vspace{-1.5ex}

    \rule{\textwidth}{0.5\fboxrule}
\setlength{\parskip}{2ex}
    Add xml text

\setlength{\parskip}{1ex}
      \textbf{Parameters}
      \vspace{-1ex}

      \begin{quote}
        \begin{Ventry}{xxx}

          \item[xml]

          xml text to send

        \end{Ventry}

      \end{quote}

    \end{boxedminipage}

    \label{tracetool:TraceNodeEx:addBackgroundColor}
    \index{tracetool \textit{(module)}!tracetool.TraceNodeEx \textit{(class)}!tracetool.TraceNodeEx.addBackgroundColor \textit{(method)}}

    \vspace{0.5ex}

\hspace{.8\funcindent}\begin{boxedminipage}{\funcwidth}

    \raggedright \textbf{addBackgroundColor}(\textit{self}, \textit{color}, \textit{colId}={\tt -1})

    \vspace{-1.5ex}

    \rule{\textwidth}{0.5\fboxrule}
\setlength{\parskip}{2ex}
    Change background font color

\setlength{\parskip}{1ex}
      \textbf{Parameters}
      \vspace{-1ex}

      \begin{quote}
        \begin{Ventry}{xxxxx}

          \item[color]

          xml background.-1 for the default color or tuple (R,G,B)

          \item[colId]

          Column index : All columns= -1,Icon=0, Time=1, thread=2, left 
          msg=3, right msg =4 or user defined column

        \end{Ventry}

      \end{quote}

    \end{boxedminipage}

    \label{tracetool:TraceNodeEx:addFontDetail}
    \index{tracetool \textit{(module)}!tracetool.TraceNodeEx \textit{(class)}!tracetool.TraceNodeEx.addFontDetail \textit{(method)}}

    \vspace{0.5ex}

\hspace{.8\funcindent}\begin{boxedminipage}{\funcwidth}

    \raggedright \textbf{addFontDetail}(\textit{self}, \textit{colId}, \textit{bold}={\tt False}, \textit{italic}={\tt False}, \textit{color}={\tt -1}, \textit{size}={\tt 0}, \textit{fontName}={\tt \texttt{'}\texttt{}\texttt{'}})

    \vspace{-1.5ex}

    \rule{\textwidth}{0.5\fboxrule}
\setlength{\parskip}{2ex}
    Change font detail for an item in the trace

\setlength{\parskip}{1ex}
      \textbf{Parameters}
      \vspace{-1ex}

      \begin{quote}
        \begin{Ventry}{xxxxxxxx}

          \item[colId]

          Column index : All columns=-1, Icon=0, Time=1, thread=2, left 
          msg=3, right msg =4 or user defined column

          \item[bold]

          Change font to bold

          \item[italic]

          Change font to Italic

          \item[color]

          Change Color. -1 for the default color or tuple (R,G,B)

          \item[size]

          Change font size, use zero to keep normal size

          \item[fontName]

          Change font name

        \end{Ventry}

      \end{quote}

      \textbf{Return Value}
    \vspace{-1ex}

      \begin{quote}
      The trace node

      \end{quote}

    \end{boxedminipage}

    \label{tracetool:TraceNodeEx:send}
    \index{tracetool \textit{(module)}!tracetool.TraceNodeEx \textit{(class)}!tracetool.TraceNodeEx.send \textit{(method)}}

    \vspace{0.5ex}

\hspace{.8\funcindent}\begin{boxedminipage}{\funcwidth}

    \raggedright \textbf{send}(\textit{self})

    \vspace{-1.5ex}

    \rule{\textwidth}{0.5\fboxrule}
\setlength{\parskip}{2ex}
    Send the traceNodeEx to the viewer (left + right + members)

\setlength{\parskip}{1ex}
      \textbf{Return Value}
    \vspace{-1ex}

      \begin{quote}
      a traceNode

      \end{quote}

    \end{boxedminipage}

    \label{tracetool:TraceNodeEx:resend}
    \index{tracetool \textit{(module)}!tracetool.TraceNodeEx \textit{(class)}!tracetool.TraceNodeEx.resend \textit{(method)}}

    \vspace{0.5ex}

\hspace{.8\funcindent}\begin{boxedminipage}{\funcwidth}

    \raggedright \textbf{resend}(\textit{self})

    \vspace{-1.5ex}

    \rule{\textwidth}{0.5\fboxrule}
\setlength{\parskip}{2ex}
    Re-send the trace to the viewer (only left and right message)

\setlength{\parskip}{1ex}
    \end{boxedminipage}


\large{\textbf{\textit{Inherited from object}}}

\begin{quote}
\_\_delattr\_\_(), \_\_format\_\_(), \_\_getattribute\_\_(), \_\_hash\_\_(), \_\_new\_\_(), \_\_reduce\_\_(), \_\_reduce\_ex\_\_(), \_\_repr\_\_(), \_\_setattr\_\_(), \_\_sizeof\_\_(), \_\_str\_\_(), \_\_subclasshook\_\_()
\end{quote}

%%%%%%%%%%%%%%%%%%%%%%%%%%%%%%%%%%%%%%%%%%%%%%%%%%%%%%%%%%%%%%%%%%%%%%%%%%%
%%                              Properties                               %%
%%%%%%%%%%%%%%%%%%%%%%%%%%%%%%%%%%%%%%%%%%%%%%%%%%%%%%%%%%%%%%%%%%%%%%%%%%%

  \subsubsection{Properties}

    \vspace{-1cm}
\hspace{\varindent}\begin{longtable}{|p{\varnamewidth}|p{\vardescrwidth}|l}
\cline{1-2}
\cline{1-2} \centering \textbf{Name} & \centering \textbf{Description}& \\
\cline{1-2}
\endhead\cline{1-2}\multicolumn{3}{r}{\small\textit{continued on next page}}\\\endfoot\cline{1-2}
\endlastfoot\multicolumn{2}{|l|}{\textit{Inherited from object}}\\
\multicolumn{2}{|p{\varwidth}|}{\raggedright \_\_class\_\_}\\
\cline{1-2}
\end{longtable}


%%%%%%%%%%%%%%%%%%%%%%%%%%%%%%%%%%%%%%%%%%%%%%%%%%%%%%%%%%%%%%%%%%%%%%%%%%%
%%                          Instance Variables                           %%
%%%%%%%%%%%%%%%%%%%%%%%%%%%%%%%%%%%%%%%%%%%%%%%%%%%%%%%%%%%%%%%%%%%%%%%%%%%

  \subsubsection{Instance Variables}

    \vspace{-1cm}
\hspace{\varindent}\begin{longtable}{|p{\varnamewidth}|p{\vardescrwidth}|l}
\cline{1-2}
\cline{1-2} \centering \textbf{Name} & \centering \textbf{Description}& \\
\cline{1-2}
\endhead\cline{1-2}\multicolumn{3}{r}{\small\textit{continued on next page}}\\\endfoot\cline{1-2}
\endlastfoot\multicolumn{2}{|l|}{\textit{Inherited from tracetool.TraceNodeBase \textit{(Section \ref{tracetool:TraceNodeBase})}}}\\
\multicolumn{2}{|p{\varwidth}|}{\raggedright enabled, iconIndex, id, tag, winTraceId}\\
\cline{1-2}
\end{longtable}

    \index{tracetool \textit{(module)}!tracetool.TraceNodeEx \textit{(class)}|)}

%%%%%%%%%%%%%%%%%%%%%%%%%%%%%%%%%%%%%%%%%%%%%%%%%%%%%%%%%%%%%%%%%%%%%%%%%%%
%%                           Class Description                           %%
%%%%%%%%%%%%%%%%%%%%%%%%%%%%%%%%%%%%%%%%%%%%%%%%%%%%%%%%%%%%%%%%%%%%%%%%%%%

    \index{tracetool \textit{(module)}!tracetool.WinWatch \textit{(class)}|(}
\subsection{Class WinWatch}

    \label{tracetool:WinWatch}
\begin{tabular}{cccccc}
% Line for object, linespec=[False]
\multicolumn{2}{r}{\settowidth{\BCL}{object}\multirow{2}{\BCL}{object}}
&&
  \\\cline{3-3}
  &&\multicolumn{1}{c|}{}
&&
  \\
&&\multicolumn{2}{l}{\textbf{tracetool.WinWatch}}
\end{tabular}

WinWatch represent a windows tree where you put watches Sample code : 
{\textgreater}{\textgreater}{\textgreater} TTrace.watches().send("test2", 
mySet)


%%%%%%%%%%%%%%%%%%%%%%%%%%%%%%%%%%%%%%%%%%%%%%%%%%%%%%%%%%%%%%%%%%%%%%%%%%%
%%                                Methods                                %%
%%%%%%%%%%%%%%%%%%%%%%%%%%%%%%%%%%%%%%%%%%%%%%%%%%%%%%%%%%%%%%%%%%%%%%%%%%%

  \subsubsection{Methods}

    \vspace{0.5ex}

\hspace{.8\funcindent}\begin{boxedminipage}{\funcwidth}

    \raggedright \textbf{\_\_init\_\_}(\textit{self}, \textit{winWatchID}={\tt None}, \textit{winWatchText}={\tt None})

    \vspace{-1.5ex}

    \rule{\textwidth}{0.5\fboxrule}
\setlength{\parskip}{2ex}
    WinWatch constructor you can map a WinWatch to an existing window. 
    Nothing Is send to the viewer if no param is given else the Window 
    watch is create on the viewer (if not already done)

\setlength{\parskip}{1ex}
      \textbf{Parameters}
      \vspace{-1ex}

      \begin{quote}
        \begin{Ventry}{xxxxxxxxxxxx}

          \item[winWatchID]

          Required window trace Id. If empty, a guid will be generated

          \item[winWatchText]

          The Window Title on the viewer.If empty, a default name will be 
          used

        \end{Ventry}

      \end{quote}

      Overrides: object.\_\_init\_\_

    \end{boxedminipage}

    \label{tracetool:WinWatch:displayWin}
    \index{tracetool \textit{(module)}!tracetool.WinWatch \textit{(class)}!tracetool.WinWatch.displayWin \textit{(method)}}

    \vspace{0.5ex}

\hspace{.8\funcindent}\begin{boxedminipage}{\funcwidth}

    \raggedright \textbf{displayWin}(\textit{self})

    \vspace{-1.5ex}

    \rule{\textwidth}{0.5\fboxrule}
\setlength{\parskip}{2ex}
    Switch viewer to this window

\setlength{\parskip}{1ex}
    \end{boxedminipage}

    \label{tracetool:WinWatch:clearAll}
    \index{tracetool \textit{(module)}!tracetool.WinWatch \textit{(class)}!tracetool.WinWatch.clearAll \textit{(method)}}

    \vspace{0.5ex}

\hspace{.8\funcindent}\begin{boxedminipage}{\funcwidth}

    \raggedright \textbf{clearAll}(\textit{self})

    \vspace{-1.5ex}

    \rule{\textwidth}{0.5\fboxrule}
\setlength{\parskip}{2ex}
    clear all traces in that window

\setlength{\parskip}{1ex}
    \end{boxedminipage}

    \label{tracetool:WinWatch:close}
    \index{tracetool \textit{(module)}!tracetool.WinWatch \textit{(class)}!tracetool.WinWatch.close \textit{(method)}}

    \vspace{0.5ex}

\hspace{.8\funcindent}\begin{boxedminipage}{\funcwidth}

    \raggedright \textbf{close}(\textit{self})

    \vspace{-1.5ex}

    \rule{\textwidth}{0.5\fboxrule}
\setlength{\parskip}{2ex}
    clear all traces in that window

\setlength{\parskip}{1ex}
    \end{boxedminipage}

    \label{tracetool:WinWatch:send}
    \index{tracetool \textit{(module)}!tracetool.WinWatch \textit{(class)}!tracetool.WinWatch.send \textit{(method)}}

    \vspace{0.5ex}

\hspace{.8\funcindent}\begin{boxedminipage}{\funcwidth}

    \raggedright \textbf{send}(\textit{self}, \textit{watchName}, \textit{watchValue})

    \vspace{-1.5ex}

    \rule{\textwidth}{0.5\fboxrule}
\setlength{\parskip}{2ex}
    Send a watch (displayed as sendValue does)

\setlength{\parskip}{1ex}
      \textbf{Parameters}
      \vspace{-1ex}

      \begin{quote}
        \begin{Ventry}{xxxxxxxxxx}

          \item[watchName]

          Watch name

          \item[watchValue]

          Watch value

        \end{Ventry}

      \end{quote}

    \end{boxedminipage}


\large{\textbf{\textit{Inherited from object}}}

\begin{quote}
\_\_delattr\_\_(), \_\_format\_\_(), \_\_getattribute\_\_(), \_\_hash\_\_(), \_\_new\_\_(), \_\_reduce\_\_(), \_\_reduce\_ex\_\_(), \_\_repr\_\_(), \_\_setattr\_\_(), \_\_sizeof\_\_(), \_\_str\_\_(), \_\_subclasshook\_\_()
\end{quote}

%%%%%%%%%%%%%%%%%%%%%%%%%%%%%%%%%%%%%%%%%%%%%%%%%%%%%%%%%%%%%%%%%%%%%%%%%%%
%%                              Properties                               %%
%%%%%%%%%%%%%%%%%%%%%%%%%%%%%%%%%%%%%%%%%%%%%%%%%%%%%%%%%%%%%%%%%%%%%%%%%%%

  \subsubsection{Properties}

    \vspace{-1cm}
\hspace{\varindent}\begin{longtable}{|p{\varnamewidth}|p{\vardescrwidth}|l}
\cline{1-2}
\cline{1-2} \centering \textbf{Name} & \centering \textbf{Description}& \\
\cline{1-2}
\endhead\cline{1-2}\multicolumn{3}{r}{\small\textit{continued on next page}}\\\endfoot\cline{1-2}
\endlastfoot\multicolumn{2}{|l|}{\textit{Inherited from object}}\\
\multicolumn{2}{|p{\varwidth}|}{\raggedright \_\_class\_\_}\\
\cline{1-2}
\end{longtable}

    \index{tracetool \textit{(module)}!tracetool.WinWatch \textit{(class)}|)}

%%%%%%%%%%%%%%%%%%%%%%%%%%%%%%%%%%%%%%%%%%%%%%%%%%%%%%%%%%%%%%%%%%%%%%%%%%%
%%                           Class Description                           %%
%%%%%%%%%%%%%%%%%%%%%%%%%%%%%%%%%%%%%%%%%%%%%%%%%%%%%%%%%%%%%%%%%%%%%%%%%%%

    \index{tracetool \textit{(module)}!tracetool.WinTrace \textit{(class)}|(}
\subsection{Class WinTrace}

    \label{tracetool:WinTrace}
\begin{tabular}{cccccccccc}
% Line for object, linespec=[False, False, False]
\multicolumn{2}{r}{\settowidth{\BCL}{object}\multirow{2}{\BCL}{object}}
&&
&&
&&
  \\\cline{3-3}
  &&\multicolumn{1}{c|}{}
&&
&&
&&
  \\
% Line for tracetool.TraceNodeBase, linespec=[False, False]
\multicolumn{4}{r}{\settowidth{\BCL}{tracetool.TraceNodeBase}\multirow{2}{\BCL}{tracetool.TraceNodeBase}}
&&
&&
  \\\cline{5-5}
  &&&&\multicolumn{1}{c|}{}
&&
&&
  \\
% Line for tracetool.TraceToSend, linespec=[False]
\multicolumn{6}{r}{\settowidth{\BCL}{tracetool.TraceToSend}\multirow{2}{\BCL}{tracetool.TraceToSend}}
&&
  \\\cline{7-7}
  &&&&&&\multicolumn{1}{c|}{}
&&
  \\
&&&&&&\multicolumn{2}{l}{\textbf{tracetool.WinTrace}}
\end{tabular}

WinTrace represent a windows tree where you put traces

Sample code :

\begin{alltt}
\pysrcprompt{{\textgreater}{\textgreater}{\textgreater} }WinTrace myWinTrace 
\pysrcprompt{{\textgreater}{\textgreater}{\textgreater} }myWinTrace = new WinTrace (\pysrcstring{"MyWINID"} , \pysrcstring{"My trace window"})
\pysrcprompt{{\textgreater}{\textgreater}{\textgreater} }myWinTrace.debug().send (\pysrcstring{"Hello"}, \pysrcstring{"Can be used to store exceptions, for examples"})\end{alltt}

%%%%%%%%%%%%%%%%%%%%%%%%%%%%%%%%%%%%%%%%%%%%%%%%%%%%%%%%%%%%%%%%%%%%%%%%%%%
%%                                Methods                                %%
%%%%%%%%%%%%%%%%%%%%%%%%%%%%%%%%%%%%%%%%%%%%%%%%%%%%%%%%%%%%%%%%%%%%%%%%%%%

  \subsubsection{Methods}

    \vspace{0.5ex}

\hspace{.8\funcindent}\begin{boxedminipage}{\funcwidth}

    \raggedright \textbf{\_\_init\_\_}(\textit{self}, \textit{winTraceID}={\tt None}, \textit{winTraceTitle}={\tt \texttt{'}\texttt{}\texttt{'}})

    \vspace{-1.5ex}

    \rule{\textwidth}{0.5\fboxrule}
\setlength{\parskip}{2ex}
    WinTrace constructor. The Window Trace is create on the viewer (if not 
    already done)

\setlength{\parskip}{1ex}
      \textbf{Parameters}
      \vspace{-1ex}

      \begin{quote}
        \begin{Ventry}{xxxxxxxxxxxxx}

          \item[winTraceID]

          Required window trace Id. If empty, a guid will be generated

          \item[winTraceTitle]

          The Window Title on the viewer.If empty, a default name will be 
          used

        \end{Ventry}

      \end{quote}

      Overrides: object.\_\_init\_\_

    \end{boxedminipage}

    \label{tracetool:WinTrace:clearAll}
    \index{tracetool \textit{(module)}!tracetool.WinTrace \textit{(class)}!tracetool.WinTrace.clearAll \textit{(method)}}

    \vspace{0.5ex}

\hspace{.8\funcindent}\begin{boxedminipage}{\funcwidth}

    \raggedright \textbf{clearAll}(\textit{self})

    \vspace{-1.5ex}

    \rule{\textwidth}{0.5\fboxrule}
\setlength{\parskip}{2ex}
    clear all trace in that window

\setlength{\parskip}{1ex}
    \end{boxedminipage}

    \label{tracetool:WinTrace:setLogFile}
    \index{tracetool \textit{(module)}!tracetool.WinTrace \textit{(class)}!tracetool.WinTrace.setLogFile \textit{(method)}}

    \vspace{0.5ex}

\hspace{.8\funcindent}\begin{boxedminipage}{\funcwidth}

    \raggedright \textbf{setLogFile}(\textit{self}, \textit{fileName}, \textit{mode}, \textit{maxLines}={\tt -1})

    \vspace{-1.5ex}

    \rule{\textwidth}{0.5\fboxrule}
\setlength{\parskip}{2ex}
    * Set the log file.(Path is relative to the viewer) * To enabled log on
    local AND on the viewer call this function twice. * To don't use the 
    viewer, set the TTrace.options.sendMode to SendMode.None.

\setlength{\parskip}{1ex}
      \textbf{Parameters}
      \vspace{-1ex}

      \begin{quote}
        \begin{Ventry}{xxxxxxxx}

          \item[fileName]

          file to save.

          \item[mode]

          Local and viewer site log mode. *  0, Viewer Log is disabled. *  
          1, Viewer log enabled. *  2, Viewer log enabled. A new file is 
          create each day (CCYYMMDD is appended to the filename) *  3, 
          Local log is disabled *  4, Local log enabled. *  5, Local log 
          enabled. A new file is create each day (CCYYMMDD is appended to 
          the filename)

          \item[maxLines]

          Number of lines before starting a new file (default : -1 = 
          unlimited)

        \end{Ventry}

      \end{quote}

    \end{boxedminipage}

    \label{tracetool:WinTrace:setMultiColumn}
    \index{tracetool \textit{(module)}!tracetool.WinTrace \textit{(class)}!tracetool.WinTrace.setMultiColumn \textit{(method)}}

    \vspace{0.5ex}

\hspace{.8\funcindent}\begin{boxedminipage}{\funcwidth}

    \raggedright \textbf{setMultiColumn}(\textit{self}, \textit{mainColIndex}={\tt 0})

    \vspace{-1.5ex}

    \rule{\textwidth}{0.5\fboxrule}
\setlength{\parskip}{2ex}
    change the tree to display user defined multiple columns

\setlength{\parskip}{1ex}
      \textbf{Parameters}
      \vspace{-1ex}

      \begin{quote}
        \begin{Ventry}{xxxxxxxxxxxx}

          \item[mainColIndex]

          The Main column index (default is 0)

        \end{Ventry}

      \end{quote}

    \end{boxedminipage}

    \label{tracetool:WinTrace:setColumnsTitle}
    \index{tracetool \textit{(module)}!tracetool.WinTrace \textit{(class)}!tracetool.WinTrace.setColumnsTitle \textit{(method)}}

    \vspace{0.5ex}

\hspace{.8\funcindent}\begin{boxedminipage}{\funcwidth}

    \raggedright \textbf{setColumnsTitle}(\textit{self}, \textit{value})

    \vspace{-1.5ex}

    \rule{\textwidth}{0.5\fboxrule}
\setlength{\parskip}{2ex}
    set columns title

\setlength{\parskip}{1ex}
      \textbf{Parameters}
      \vspace{-1ex}

      \begin{quote}
        \begin{Ventry}{xxxxx}

          \item[value]

          tab separated columns titles

        \end{Ventry}

      \end{quote}

    \end{boxedminipage}

    \label{tracetool:WinTrace:setColumnsWidth}
    \index{tracetool \textit{(module)}!tracetool.WinTrace \textit{(class)}!tracetool.WinTrace.setColumnsWidth \textit{(method)}}

    \vspace{0.5ex}

\hspace{.8\funcindent}\begin{boxedminipage}{\funcwidth}

    \raggedright \textbf{setColumnsWidth}(\textit{self}, \textit{widths})

    \vspace{-1.5ex}

    \rule{\textwidth}{0.5\fboxrule}
\setlength{\parskip}{2ex}
    set columns widths

\setlength{\parskip}{1ex}
      \textbf{Parameters}
      \vspace{-1ex}

      \begin{quote}
        \begin{Ventry}{xxxxxx}

          \item[widths]

          Tab separated columns width. The format for each column is 
          width[:Min[:Max]] where Min and Max are optional minimum and 
          maximum column width for resizing purpose. Example : 100:20:80 
          tab 200:50 tab 100

        \end{Ventry}

      \end{quote}

    \end{boxedminipage}

    \label{tracetool:WinTrace:gotoFirstNode}
    \index{tracetool \textit{(module)}!tracetool.WinTrace \textit{(class)}!tracetool.WinTrace.gotoFirstNode \textit{(method)}}

    \vspace{0.5ex}

\hspace{.8\funcindent}\begin{boxedminipage}{\funcwidth}

    \raggedright \textbf{gotoFirstNode}(\textit{self})

    \vspace{-1.5ex}

    \rule{\textwidth}{0.5\fboxrule}
\setlength{\parskip}{2ex}
    Set the focus to the trace first node

\setlength{\parskip}{1ex}
    \end{boxedminipage}

    \label{tracetool:WinTrace:gotoLastNode}
    \index{tracetool \textit{(module)}!tracetool.WinTrace \textit{(class)}!tracetool.WinTrace.gotoLastNode \textit{(method)}}

    \vspace{0.5ex}

\hspace{.8\funcindent}\begin{boxedminipage}{\funcwidth}

    \raggedright \textbf{gotoLastNode}(\textit{self})

    \vspace{-1.5ex}

    \rule{\textwidth}{0.5\fboxrule}
\setlength{\parskip}{2ex}
    Set the focus to the trace last node

\setlength{\parskip}{1ex}
    \end{boxedminipage}

    \label{tracetool:WinTrace:findNext}
    \index{tracetool \textit{(module)}!tracetool.WinTrace \textit{(class)}!tracetool.WinTrace.findNext \textit{(method)}}

    \vspace{0.5ex}

\hspace{.8\funcindent}\begin{boxedminipage}{\funcwidth}

    \raggedright \textbf{findNext}(\textit{self}, \textit{searchForward}={\tt True})

    \vspace{-1.5ex}

    \rule{\textwidth}{0.5\fboxrule}
\setlength{\parskip}{2ex}
    Set the focus to the next matching node

\setlength{\parskip}{1ex}
      \textbf{Parameters}
      \vspace{-1ex}

      \begin{quote}
        \begin{Ventry}{xxxxxxxxxxxxx}

          \item[searchForward]

          If true search down, else search up

        \end{Ventry}

      \end{quote}

    \end{boxedminipage}

    \label{tracetool:WinTrace:gotoBookmark}
    \index{tracetool \textit{(module)}!tracetool.WinTrace \textit{(class)}!tracetool.WinTrace.gotoBookmark \textit{(method)}}

    \vspace{0.5ex}

\hspace{.8\funcindent}\begin{boxedminipage}{\funcwidth}

    \raggedright \textbf{gotoBookmark}(\textit{self}, \textit{pos})

    \vspace{-1.5ex}

    \rule{\textwidth}{0.5\fboxrule}
\setlength{\parskip}{2ex}
    Set the focus to a bookmarked node identified by his position. 
    Bookmarks are cheched by the user or with the node.SetBookmark() 
    function

\setlength{\parskip}{1ex}
      \textbf{Parameters}
      \vspace{-1ex}

      \begin{quote}
        \begin{Ventry}{xxx}

          \item[pos]

          Indice of the bookmark

        \end{Ventry}

      \end{quote}

    \end{boxedminipage}

    \label{tracetool:WinTrace:clearBookmark}
    \index{tracetool \textit{(module)}!tracetool.WinTrace \textit{(class)}!tracetool.WinTrace.clearBookmark \textit{(method)}}

    \vspace{0.5ex}

\hspace{.8\funcindent}\begin{boxedminipage}{\funcwidth}

    \raggedright \textbf{clearBookmark}(\textit{self})

    \vspace{-1.5ex}

    \rule{\textwidth}{0.5\fboxrule}
\setlength{\parskip}{2ex}
    Clear all bookmarks

\setlength{\parskip}{1ex}
    \end{boxedminipage}

    \label{tracetool:WinTrace:clearFilter}
    \index{tracetool \textit{(module)}!tracetool.WinTrace \textit{(class)}!tracetool.WinTrace.clearFilter \textit{(method)}}

    \vspace{0.5ex}

\hspace{.8\funcindent}\begin{boxedminipage}{\funcwidth}

    \raggedright \textbf{clearFilter}(\textit{self})

    \vspace{-1.5ex}

    \rule{\textwidth}{0.5\fboxrule}
\setlength{\parskip}{2ex}
    Clear all filters

\setlength{\parskip}{1ex}
    \end{boxedminipage}

    \label{tracetool:WinTrace:addFilter}
    \index{tracetool \textit{(module)}!tracetool.WinTrace \textit{(class)}!tracetool.WinTrace.addFilter \textit{(method)}}

    \vspace{0.5ex}

\hspace{.8\funcindent}\begin{boxedminipage}{\funcwidth}

    \raggedright \textbf{addFilter}(\textit{self}, \textit{column}, \textit{compare}, \textit{text})

    \vspace{-1.5ex}

    \rule{\textwidth}{0.5\fboxrule}
\setlength{\parskip}{2ex}
    Add a filter to node. Multiple calls to this function can be done. Call
    ApplyFilter() to apply filtering

\setlength{\parskip}{1ex}
      \textbf{Parameters}
      \vspace{-1ex}

      \begin{quote}
        \begin{Ventry}{xxxxxxx}

          \item[column]

          column to apply filter. In multicolumn mode the first column 
          start at 0 In normal mode : col icone   = 999 col time    = 1 col
          thread  = 2 col traces  = 3 col Comment = 4 col members = 998

          \item[compare]

          There is 5 kinds of filters : Equal           = 0 Not equal
          = 1 Contains        = 2 Don't contains  = 3 (Ignore this filter) 
          = 4 or -1

          \item[text]

          The text to search (insensitive)

        \end{Ventry}

      \end{quote}

    \end{boxedminipage}

    \label{tracetool:WinTrace:applyFilter}
    \index{tracetool \textit{(module)}!tracetool.WinTrace \textit{(class)}!tracetool.WinTrace.applyFilter \textit{(method)}}

    \vspace{0.5ex}

\hspace{.8\funcindent}\begin{boxedminipage}{\funcwidth}

    \raggedright \textbf{applyFilter}(\textit{self}, \textit{conditionAnd}, \textit{showMatch}, \textit{includeChildren})

    \vspace{-1.5ex}

    \rule{\textwidth}{0.5\fboxrule}
\setlength{\parskip}{2ex}
    Apply filters after calls to AddFilter().

\setlength{\parskip}{1ex}
      \textbf{Parameters}
      \vspace{-1ex}

      \begin{quote}
        \begin{Ventry}{xxxxxxxxxxxxxxx}

          \item[conditionAnd]

          If true, use an 'AND' condition for each filters, else use a "OR"

          \item[showMatch]

          If true, show node that match filter and hide others. If false 
          hide matching node and show others

          \item[includeChildren]

          If true, search in subnodes

        \end{Ventry}

      \end{quote}

    \end{boxedminipage}

    \label{tracetool:WinTrace:saveToTextfile}
    \index{tracetool \textit{(module)}!tracetool.WinTrace \textit{(class)}!tracetool.WinTrace.saveToTextfile \textit{(method)}}

    \vspace{0.5ex}

\hspace{.8\funcindent}\begin{boxedminipage}{\funcwidth}

    \raggedright \textbf{saveToTextfile}(\textit{self}, \textit{fileName})

    \vspace{-1.5ex}

    \rule{\textwidth}{0.5\fboxrule}
\setlength{\parskip}{2ex}
    Save window content to text file  (Path is relative to the viewer)

\setlength{\parskip}{1ex}
      \textbf{Parameters}
      \vspace{-1ex}

      \begin{quote}
        \begin{Ventry}{xxxxxxxx}

          \item[fileName]

          target filename

        \end{Ventry}

      \end{quote}

    \end{boxedminipage}

    \label{tracetool:WinTrace:saveToXml}
    \index{tracetool \textit{(module)}!tracetool.WinTrace \textit{(class)}!tracetool.WinTrace.saveToXml \textit{(method)}}

    \vspace{0.5ex}

\hspace{.8\funcindent}\begin{boxedminipage}{\funcwidth}

    \raggedright \textbf{saveToXml}(\textit{self}, \textit{fileName}, \textit{styleSheet}={\tt \texttt{'}\texttt{}\texttt{'}})

    \vspace{-1.5ex}

    \rule{\textwidth}{0.5\fboxrule}
\setlength{\parskip}{2ex}
    Save window content to xml file (Path is relative to the viewer)

\setlength{\parskip}{1ex}
      \textbf{Parameters}
      \vspace{-1ex}

      \begin{quote}
        \begin{Ventry}{xxxxxxxxxx}

          \item[fileName]

          target filename

          \item[styleSheet]

          optional stylesheet file name

        \end{Ventry}

      \end{quote}

    \end{boxedminipage}

    \label{tracetool:WinTrace:loadXml}
    \index{tracetool \textit{(module)}!tracetool.WinTrace \textit{(class)}!tracetool.WinTrace.loadXml \textit{(method)}}

    \vspace{0.5ex}

\hspace{.8\funcindent}\begin{boxedminipage}{\funcwidth}

    \raggedright \textbf{loadXml}(\textit{self}, \textit{fileName})

    \vspace{-1.5ex}

    \rule{\textwidth}{0.5\fboxrule}
\setlength{\parskip}{2ex}
    Load xml file to the window (Path is relative to the viewer)

\setlength{\parskip}{1ex}
      \textbf{Parameters}
      \vspace{-1ex}

      \begin{quote}
        \begin{Ventry}{xxxxxxxx}

          \item[fileName]

          target filename

        \end{Ventry}

      \end{quote}

    \end{boxedminipage}

    \label{tracetool:WinTrace:displayWin}
    \index{tracetool \textit{(module)}!tracetool.WinTrace \textit{(class)}!tracetool.WinTrace.displayWin \textit{(method)}}

    \vspace{0.5ex}

\hspace{.8\funcindent}\begin{boxedminipage}{\funcwidth}

    \raggedright \textbf{displayWin}(\textit{self})

    \vspace{-1.5ex}

    \rule{\textwidth}{0.5\fboxrule}
\setlength{\parskip}{2ex}
    Switch viewer to this window

\setlength{\parskip}{1ex}
    \end{boxedminipage}

    \label{tracetool:WinTrace:close}
    \index{tracetool \textit{(module)}!tracetool.WinTrace \textit{(class)}!tracetool.WinTrace.close \textit{(method)}}

    \vspace{0.5ex}

\hspace{.8\funcindent}\begin{boxedminipage}{\funcwidth}

    \raggedright \textbf{close}(\textit{self})

    \vspace{-1.5ex}

    \rule{\textwidth}{0.5\fboxrule}
\setlength{\parskip}{2ex}
    clear all trace in that window

\setlength{\parskip}{1ex}
    \end{boxedminipage}


\large{\textbf{\textit{Inherited from tracetool.TraceToSend\textit{(Section \ref{tracetool:TraceToSend})}}}}

\begin{quote}
deleteLastContext(), enterMethod(), exitMethod(), getIndentLevel(), getLastContext(), getLastContextId(), indent(), prepareNewNode(), pushContext(), send(), sendBackgroundColor(), sendCaller(), sendDump(), sendObject(), sendStack(), sendTable(), sendValue(), sendXml(), unIndent()
\end{quote}

\large{\textbf{\textit{Inherited from object}}}

\begin{quote}
\_\_delattr\_\_(), \_\_format\_\_(), \_\_getattribute\_\_(), \_\_hash\_\_(), \_\_new\_\_(), \_\_reduce\_\_(), \_\_reduce\_ex\_\_(), \_\_repr\_\_(), \_\_setattr\_\_(), \_\_sizeof\_\_(), \_\_str\_\_(), \_\_subclasshook\_\_()
\end{quote}

%%%%%%%%%%%%%%%%%%%%%%%%%%%%%%%%%%%%%%%%%%%%%%%%%%%%%%%%%%%%%%%%%%%%%%%%%%%
%%                              Properties                               %%
%%%%%%%%%%%%%%%%%%%%%%%%%%%%%%%%%%%%%%%%%%%%%%%%%%%%%%%%%%%%%%%%%%%%%%%%%%%

  \subsubsection{Properties}

    \vspace{-1cm}
\hspace{\varindent}\begin{longtable}{|p{\varnamewidth}|p{\vardescrwidth}|l}
\cline{1-2}
\cline{1-2} \centering \textbf{Name} & \centering \textbf{Description}& \\
\cline{1-2}
\endhead\cline{1-2}\multicolumn{3}{r}{\small\textit{continued on next page}}\\\endfoot\cline{1-2}
\endlastfoot\multicolumn{2}{|l|}{\textit{Inherited from object}}\\
\multicolumn{2}{|p{\varwidth}|}{\raggedright \_\_class\_\_}\\
\cline{1-2}
\end{longtable}


%%%%%%%%%%%%%%%%%%%%%%%%%%%%%%%%%%%%%%%%%%%%%%%%%%%%%%%%%%%%%%%%%%%%%%%%%%%
%%                          Instance Variables                           %%
%%%%%%%%%%%%%%%%%%%%%%%%%%%%%%%%%%%%%%%%%%%%%%%%%%%%%%%%%%%%%%%%%%%%%%%%%%%

  \subsubsection{Instance Variables}

    \vspace{-1cm}
\hspace{\varindent}\begin{longtable}{|p{\varnamewidth}|p{\vardescrwidth}|l}
\cline{1-2}
\cline{1-2} \centering \textbf{Name} & \centering \textbf{Description}& \\
\cline{1-2}
\endhead\cline{1-2}\multicolumn{3}{r}{\small\textit{continued on next page}}\\\endfoot\cline{1-2}
\endlastfoot\multicolumn{2}{|l|}{\textit{Inherited from tracetool.TraceNodeBase \textit{(Section \ref{tracetool:TraceNodeBase})}}}\\
\multicolumn{2}{|p{\varwidth}|}{\raggedright enabled, iconIndex, id, tag, winTraceId}\\
\cline{1-2}
\end{longtable}

    \index{tracetool \textit{(module)}!tracetool.WinTrace \textit{(class)}|)}

%%%%%%%%%%%%%%%%%%%%%%%%%%%%%%%%%%%%%%%%%%%%%%%%%%%%%%%%%%%%%%%%%%%%%%%%%%%
%%                           Class Description                           %%
%%%%%%%%%%%%%%%%%%%%%%%%%%%%%%%%%%%%%%%%%%%%%%%%%%%%%%%%%%%%%%%%%%%%%%%%%%%

    \index{tracetool \textit{(module)}!tracetool.TTraceOptions \textit{(class)}|(}
\subsection{Class TTraceOptions}

    \label{tracetool:TTraceOptions}
\begin{tabular}{cccccc}
% Line for object, linespec=[False]
\multicolumn{2}{r}{\settowidth{\BCL}{object}\multirow{2}{\BCL}{object}}
&&
  \\\cline{3-3}
  &&\multicolumn{1}{c|}{}
&&
  \\
&&\multicolumn{2}{l}{\textbf{tracetool.TTraceOptions}}
\end{tabular}

Options for the traces (Host,Port,...)


%%%%%%%%%%%%%%%%%%%%%%%%%%%%%%%%%%%%%%%%%%%%%%%%%%%%%%%%%%%%%%%%%%%%%%%%%%%
%%                                Methods                                %%
%%%%%%%%%%%%%%%%%%%%%%%%%%%%%%%%%%%%%%%%%%%%%%%%%%%%%%%%%%%%%%%%%%%%%%%%%%%

  \subsubsection{Methods}

    \vspace{0.5ex}

\hspace{.8\funcindent}\begin{boxedminipage}{\funcwidth}

    \raggedright \textbf{\_\_init\_\_}(\textit{self})

\setlength{\parskip}{2ex}
    x.\_\_init\_\_(...) initializes x; see x.\_\_class\_\_.\_\_doc\_\_ for 
    signature

\setlength{\parskip}{1ex}
      Overrides: object.\_\_init\_\_ 	extit{(inherited documentation)}

    \end{boxedminipage}


\large{\textbf{\textit{Inherited from object}}}

\begin{quote}
\_\_delattr\_\_(), \_\_format\_\_(), \_\_getattribute\_\_(), \_\_hash\_\_(), \_\_new\_\_(), \_\_reduce\_\_(), \_\_reduce\_ex\_\_(), \_\_repr\_\_(), \_\_setattr\_\_(), \_\_sizeof\_\_(), \_\_str\_\_(), \_\_subclasshook\_\_()
\end{quote}

%%%%%%%%%%%%%%%%%%%%%%%%%%%%%%%%%%%%%%%%%%%%%%%%%%%%%%%%%%%%%%%%%%%%%%%%%%%
%%                              Properties                               %%
%%%%%%%%%%%%%%%%%%%%%%%%%%%%%%%%%%%%%%%%%%%%%%%%%%%%%%%%%%%%%%%%%%%%%%%%%%%

  \subsubsection{Properties}

    \vspace{-1cm}
\hspace{\varindent}\begin{longtable}{|p{\varnamewidth}|p{\vardescrwidth}|l}
\cline{1-2}
\cline{1-2} \centering \textbf{Name} & \centering \textbf{Description}& \\
\cline{1-2}
\endhead\cline{1-2}\multicolumn{3}{r}{\small\textit{continued on next page}}\\\endfoot\cline{1-2}
\endlastfoot\multicolumn{2}{|l|}{\textit{Inherited from object}}\\
\multicolumn{2}{|p{\varwidth}|}{\raggedright \_\_class\_\_}\\
\cline{1-2}
\end{longtable}

    \index{tracetool \textit{(module)}!tracetool.TTraceOptions \textit{(class)}|)}

%%%%%%%%%%%%%%%%%%%%%%%%%%%%%%%%%%%%%%%%%%%%%%%%%%%%%%%%%%%%%%%%%%%%%%%%%%%
%%                           Class Description                           %%
%%%%%%%%%%%%%%%%%%%%%%%%%%%%%%%%%%%%%%%%%%%%%%%%%%%%%%%%%%%%%%%%%%%%%%%%%%%

    \index{tracetool \textit{(module)}!tracetool.Internals \textit{(class)}|(}
\subsection{Class Internals}

    \label{tracetool:Internals}
\begin{tabular}{cccccc}
% Line for object, linespec=[False]
\multicolumn{2}{r}{\settowidth{\BCL}{object}\multirow{2}{\BCL}{object}}
&&
  \\\cline{3-3}
  &&\multicolumn{1}{c|}{}
&&
  \\
&&\multicolumn{2}{l}{\textbf{tracetool.Internals}}
\end{tabular}

container for internals (so private) tracetool vars


%%%%%%%%%%%%%%%%%%%%%%%%%%%%%%%%%%%%%%%%%%%%%%%%%%%%%%%%%%%%%%%%%%%%%%%%%%%
%%                                Methods                                %%
%%%%%%%%%%%%%%%%%%%%%%%%%%%%%%%%%%%%%%%%%%%%%%%%%%%%%%%%%%%%%%%%%%%%%%%%%%%

  \subsubsection{Methods}

    \vspace{0.5ex}

\hspace{.8\funcindent}\begin{boxedminipage}{\funcwidth}

    \raggedright \textbf{\_\_init\_\_}(\textit{self})

\setlength{\parskip}{2ex}
    x.\_\_init\_\_(...) initializes x; see x.\_\_class\_\_.\_\_doc\_\_ for 
    signature

\setlength{\parskip}{1ex}
      Overrides: object.\_\_init\_\_ 	extit{(inherited documentation)}

    \end{boxedminipage}


\large{\textbf{\textit{Inherited from object}}}

\begin{quote}
\_\_delattr\_\_(), \_\_format\_\_(), \_\_getattribute\_\_(), \_\_hash\_\_(), \_\_new\_\_(), \_\_reduce\_\_(), \_\_reduce\_ex\_\_(), \_\_repr\_\_(), \_\_setattr\_\_(), \_\_sizeof\_\_(), \_\_str\_\_(), \_\_subclasshook\_\_()
\end{quote}

%%%%%%%%%%%%%%%%%%%%%%%%%%%%%%%%%%%%%%%%%%%%%%%%%%%%%%%%%%%%%%%%%%%%%%%%%%%
%%                              Properties                               %%
%%%%%%%%%%%%%%%%%%%%%%%%%%%%%%%%%%%%%%%%%%%%%%%%%%%%%%%%%%%%%%%%%%%%%%%%%%%

  \subsubsection{Properties}

    \vspace{-1cm}
\hspace{\varindent}\begin{longtable}{|p{\varnamewidth}|p{\vardescrwidth}|l}
\cline{1-2}
\cline{1-2} \centering \textbf{Name} & \centering \textbf{Description}& \\
\cline{1-2}
\endhead\cline{1-2}\multicolumn{3}{r}{\small\textit{continued on next page}}\\\endfoot\cline{1-2}
\endlastfoot\multicolumn{2}{|l|}{\textit{Inherited from object}}\\
\multicolumn{2}{|p{\varwidth}|}{\raggedright \_\_class\_\_}\\
\cline{1-2}
\end{longtable}

    \index{tracetool \textit{(module)}!tracetool.Internals \textit{(class)}|)}

%%%%%%%%%%%%%%%%%%%%%%%%%%%%%%%%%%%%%%%%%%%%%%%%%%%%%%%%%%%%%%%%%%%%%%%%%%%
%%                           Class Description                           %%
%%%%%%%%%%%%%%%%%%%%%%%%%%%%%%%%%%%%%%%%%%%%%%%%%%%%%%%%%%%%%%%%%%%%%%%%%%%

    \index{tracetool \textit{(module)}!tracetool.TTrace \textit{(class)}|(}
\subsection{Class TTrace}

    \label{tracetool:TTrace}
\begin{tabular}{cccccc}
% Line for object, linespec=[False]
\multicolumn{2}{r}{\settowidth{\BCL}{object}\multirow{2}{\BCL}{object}}
&&
  \\\cline{3-3}
  &&\multicolumn{1}{c|}{}
&&
  \\
&&\multicolumn{2}{l}{\textbf{tracetool.TTrace}}
\end{tabular}

TTrace is the entry point for all traces. TTrace give 3 'TraceNode' doors :
Warning , Error and Debug. Theses 3 doors are displayed with a special icon
(all of them have the 'enabled' property set to true The class is fully 
static


%%%%%%%%%%%%%%%%%%%%%%%%%%%%%%%%%%%%%%%%%%%%%%%%%%%%%%%%%%%%%%%%%%%%%%%%%%%
%%                                Methods                                %%
%%%%%%%%%%%%%%%%%%%%%%%%%%%%%%%%%%%%%%%%%%%%%%%%%%%%%%%%%%%%%%%%%%%%%%%%%%%

  \subsubsection{Methods}

    \label{tracetool:TTrace:show}
    \index{tracetool \textit{(module)}!tracetool.TTrace \textit{(class)}!tracetool.TTrace.show \textit{(static method)}}

    \vspace{0.5ex}

\hspace{.8\funcindent}\begin{boxedminipage}{\funcwidth}

    \raggedright \textbf{show}(\textit{isVisible}={\tt True})

    \vspace{-1.5ex}

    \rule{\textwidth}{0.5\fboxrule}
\setlength{\parskip}{2ex}
    Show or hide the trace program

\setlength{\parskip}{1ex}
      \textbf{Parameters}
      \vspace{-1ex}

      \begin{quote}
        \begin{Ventry}{xxxxxxxxx}

          \item[isVisible]

          True to show the viewer. False to hide

        \end{Ventry}

      \end{quote}

    \end{boxedminipage}

    \label{tracetool:TTrace:find}
    \index{tracetool \textit{(module)}!tracetool.TTrace \textit{(class)}!tracetool.TTrace.find \textit{(static method)}}

    \vspace{0.5ex}

\hspace{.8\funcindent}\begin{boxedminipage}{\funcwidth}

    \raggedright \textbf{find}(\textit{Text}, \textit{Sensitive}={\tt True}, \textit{WholeWord}={\tt False}, \textit{Highlight}={\tt False}, \textit{SearchInAllPages}={\tt False})

    \vspace{-1.5ex}

    \rule{\textwidth}{0.5\fboxrule}
\setlength{\parskip}{2ex}
    Set the global search criteria. You must call 
    TTrace.winTrace.findNext() to position to the next or previous matching
    node

\setlength{\parskip}{1ex}
      \textbf{Parameters}
      \vspace{-1ex}

      \begin{quote}
        \begin{Ventry}{xxxxxxxxxxxxxxxx}

          \item[Text]

          Text to search

          \item[Sensitive]

          Search is case sensitive

          \item[WholeWord]

          Match only whole word

          \item[Highlight]

          Highlight results

          \item[SearchInAllPages]

          Call to FindNext will search also in other traces windows if true

        \end{Ventry}

      \end{quote}

    \end{boxedminipage}

    \label{tracetool:TTrace:closeViewer}
    \index{tracetool \textit{(module)}!tracetool.TTrace \textit{(class)}!tracetool.TTrace.closeViewer \textit{(static method)}}

    \vspace{0.5ex}

\hspace{.8\funcindent}\begin{boxedminipage}{\funcwidth}

    \raggedright \textbf{closeViewer}()

    \vspace{-1.5ex}

    \rule{\textwidth}{0.5\fboxrule}
\setlength{\parskip}{2ex}
    Close the viewer (not hide as show(false)

\setlength{\parskip}{1ex}
    \end{boxedminipage}

    \label{tracetool:TTrace:clearAll}
    \index{tracetool \textit{(module)}!tracetool.TTrace \textit{(class)}!tracetool.TTrace.clearAll \textit{(static method)}}

    \vspace{0.5ex}

\hspace{.8\funcindent}\begin{boxedminipage}{\funcwidth}

    \raggedright \textbf{clearAll}()

    \vspace{-1.5ex}

    \rule{\textwidth}{0.5\fboxrule}
\setlength{\parskip}{2ex}
    Clear all traces

\setlength{\parskip}{1ex}
    \end{boxedminipage}

    \label{tracetool:TTrace:stop}
    \index{tracetool \textit{(module)}!tracetool.TTrace \textit{(class)}!tracetool.TTrace.stop \textit{(static method)}}

    \vspace{0.5ex}

\hspace{.8\funcindent}\begin{boxedminipage}{\funcwidth}

    \raggedright \textbf{stop}()

    \vspace{-1.5ex}

    \rule{\textwidth}{0.5\fboxrule}
\setlength{\parskip}{2ex}
    Stop sub-system before leaving your program. You may call 
    ttrace.flush() before stop()

\setlength{\parskip}{1ex}
    \end{boxedminipage}

    \label{tracetool:TTrace:flush}
    \index{tracetool \textit{(module)}!tracetool.TTrace \textit{(class)}!tracetool.TTrace.flush \textit{(static method)}}

    \vspace{0.5ex}

\hspace{.8\funcindent}\begin{boxedminipage}{\funcwidth}

    \raggedright \textbf{flush}()

    \vspace{-1.5ex}

    \rule{\textwidth}{0.5\fboxrule}
\setlength{\parskip}{2ex}
    flush remaining traces to the viewer

\setlength{\parskip}{1ex}
    \end{boxedminipage}


\large{\textbf{\textit{Inherited from object}}}

\begin{quote}
\_\_delattr\_\_(), \_\_format\_\_(), \_\_getattribute\_\_(), \_\_hash\_\_(), \_\_init\_\_(), \_\_new\_\_(), \_\_reduce\_\_(), \_\_reduce\_ex\_\_(), \_\_repr\_\_(), \_\_setattr\_\_(), \_\_sizeof\_\_(), \_\_str\_\_(), \_\_subclasshook\_\_()
\end{quote}

%%%%%%%%%%%%%%%%%%%%%%%%%%%%%%%%%%%%%%%%%%%%%%%%%%%%%%%%%%%%%%%%%%%%%%%%%%%
%%                              Properties                               %%
%%%%%%%%%%%%%%%%%%%%%%%%%%%%%%%%%%%%%%%%%%%%%%%%%%%%%%%%%%%%%%%%%%%%%%%%%%%

  \subsubsection{Properties}

    \vspace{-1cm}
\hspace{\varindent}\begin{longtable}{|p{\varnamewidth}|p{\vardescrwidth}|l}
\cline{1-2}
\cline{1-2} \centering \textbf{Name} & \centering \textbf{Description}& \\
\cline{1-2}
\endhead\cline{1-2}\multicolumn{3}{r}{\small\textit{continued on next page}}\\\endfoot\cline{1-2}
\endlastfoot\multicolumn{2}{|l|}{\textit{Inherited from object}}\\
\multicolumn{2}{|p{\varwidth}|}{\raggedright \_\_class\_\_}\\
\cline{1-2}
\end{longtable}


%%%%%%%%%%%%%%%%%%%%%%%%%%%%%%%%%%%%%%%%%%%%%%%%%%%%%%%%%%%%%%%%%%%%%%%%%%%
%%                            Class Variables                            %%
%%%%%%%%%%%%%%%%%%%%%%%%%%%%%%%%%%%%%%%%%%%%%%%%%%%%%%%%%%%%%%%%%%%%%%%%%%%

  \subsubsection{Class Variables}

    \vspace{-1cm}
\hspace{\varindent}\begin{longtable}{|p{\varnamewidth}|p{\vardescrwidth}|l}
\cline{1-2}
\cline{1-2} \centering \textbf{Name} & \centering \textbf{Description}& \\
\cline{1-2}
\endhead\cline{1-2}\multicolumn{3}{r}{\small\textit{continued on next page}}\\\endfoot\cline{1-2}
\endlastfoot\raggedright n\-a\-t\-i\-v\-e\-C\-l\-a\-s\-s\-e\-s\- & \raggedright \textbf{Value:} 
{\tt \texttt{[}\texttt{]}}&\\
\cline{1-2}
\raggedright R\-g\-b\-C\-o\-l\-o\-r\-s\- & \raggedright \textbf{Value:} 
{\tt \texttt{\{}\texttt{'}\texttt{Aqua}\texttt{'}\texttt{: }\texttt{(}0\texttt{, }255\texttt{, }255\texttt{)}\texttt{, }\texttt{'}\texttt{Black}\texttt{'}\texttt{: }\texttt{(}0\texttt{, }0\texttt{, }0\texttt{)}\texttt{, }\texttt{'}\texttt{Blue}\texttt{'}\texttt{: }\texttt{(}0\texttt{, }0\texttt{...}}&\\
\cline{1-2}
\raggedright T\-r\-a\-c\-e\-C\-o\-n\-s\-t\- & \raggedright \textbf{Value:} 
{\tt \texttt{\{}\texttt{'}\texttt{CST\_ACTION\_CLEAR\_ALL}\texttt{'}\texttt{: }10\texttt{, }\texttt{'}\texttt{CST\_ACTION\_CLOSE\_WIN}\texttt{'}\texttt{: }11\texttt{, }\texttt{...}}&\\
\cline{1-2}
\raggedright o\-p\-t\-i\-o\-n\-s\- & \raggedright \textbf{Value:} 
{\tt TTraceOptions()}&\\
\cline{1-2}
\raggedright i\-n\-t\-e\-r\-n\-a\-l\-s\- & \raggedright \textbf{Value:} 
{\tt Internals()}&\\
\cline{1-2}
\raggedright w\-i\-n\-T\-r\-a\-c\-e\- & \raggedright \textbf{Value:} 
{\tt WinTrace("\_")}&\\
\cline{1-2}
\raggedright w\-a\-t\-c\-h\-e\-s\- & \raggedright \textbf{Value:} 
{\tt WinWatch()}&\\
\cline{1-2}
\raggedright d\-e\-b\-u\-g\- & \raggedright \textbf{Value:} 
{\tt TTrace.winTrace.debug}&\\
\cline{1-2}
\raggedright w\-a\-r\-n\-i\-n\-g\- & \raggedright \textbf{Value:} 
{\tt TTrace.winTrace.warning}&\\
\cline{1-2}
\raggedright e\-r\-r\-o\-r\- & \raggedright \textbf{Value:} 
{\tt TTrace.winTrace.error}&\\
\cline{1-2}
\end{longtable}

    \index{tracetool \textit{(module)}!tracetool.TTrace \textit{(class)}|)}

%%%%%%%%%%%%%%%%%%%%%%%%%%%%%%%%%%%%%%%%%%%%%%%%%%%%%%%%%%%%%%%%%%%%%%%%%%%
%%                           Class Description                           %%
%%%%%%%%%%%%%%%%%%%%%%%%%%%%%%%%%%%%%%%%%%%%%%%%%%%%%%%%%%%%%%%%%%%%%%%%%%%

    \index{tracetool \textit{(module)}!tracetool.TTrace \textit{(class)}|(}
\subsection{Class TTrace}

    \label{tracetool:TTrace}
\begin{tabular}{cccccc}
% Line for object, linespec=[False]
\multicolumn{2}{r}{\settowidth{\BCL}{object}\multirow{2}{\BCL}{object}}
&&
  \\\cline{3-3}
  &&\multicolumn{1}{c|}{}
&&
  \\
&&\multicolumn{2}{l}{\textbf{tracetool.TTrace}}
\end{tabular}

TTrace is the entry point for all traces. TTrace give 3 'TraceNode' doors :
Warning , Error and Debug. Theses 3 doors are displayed with a special icon
(all of them have the 'enabled' property set to true The class is fully 
static


%%%%%%%%%%%%%%%%%%%%%%%%%%%%%%%%%%%%%%%%%%%%%%%%%%%%%%%%%%%%%%%%%%%%%%%%%%%
%%                                Methods                                %%
%%%%%%%%%%%%%%%%%%%%%%%%%%%%%%%%%%%%%%%%%%%%%%%%%%%%%%%%%%%%%%%%%%%%%%%%%%%

  \subsubsection{Methods}

    \label{tracetool:TTrace:show}
    \index{tracetool \textit{(module)}!tracetool.TTrace \textit{(class)}!tracetool.TTrace.show \textit{(static method)}}

    \vspace{0.5ex}

\hspace{.8\funcindent}\begin{boxedminipage}{\funcwidth}

    \raggedright \textbf{show}(\textit{isVisible}={\tt True})

    \vspace{-1.5ex}

    \rule{\textwidth}{0.5\fboxrule}
\setlength{\parskip}{2ex}
    Show or hide the trace program

\setlength{\parskip}{1ex}
      \textbf{Parameters}
      \vspace{-1ex}

      \begin{quote}
        \begin{Ventry}{xxxxxxxxx}

          \item[isVisible]

          True to show the viewer. False to hide

        \end{Ventry}

      \end{quote}

    \end{boxedminipage}

    \label{tracetool:TTrace:find}
    \index{tracetool \textit{(module)}!tracetool.TTrace \textit{(class)}!tracetool.TTrace.find \textit{(static method)}}

    \vspace{0.5ex}

\hspace{.8\funcindent}\begin{boxedminipage}{\funcwidth}

    \raggedright \textbf{find}(\textit{Text}, \textit{Sensitive}={\tt True}, \textit{WholeWord}={\tt False}, \textit{Highlight}={\tt False}, \textit{SearchInAllPages}={\tt False})

    \vspace{-1.5ex}

    \rule{\textwidth}{0.5\fboxrule}
\setlength{\parskip}{2ex}
    Set the global search criteria. You must call 
    TTrace.winTrace.findNext() to position to the next or previous matching
    node

\setlength{\parskip}{1ex}
      \textbf{Parameters}
      \vspace{-1ex}

      \begin{quote}
        \begin{Ventry}{xxxxxxxxxxxxxxxx}

          \item[Text]

          Text to search

          \item[Sensitive]

          Search is case sensitive

          \item[WholeWord]

          Match only whole word

          \item[Highlight]

          Highlight results

          \item[SearchInAllPages]

          Call to FindNext will search also in other traces windows if true

        \end{Ventry}

      \end{quote}

    \end{boxedminipage}

    \label{tracetool:TTrace:closeViewer}
    \index{tracetool \textit{(module)}!tracetool.TTrace \textit{(class)}!tracetool.TTrace.closeViewer \textit{(static method)}}

    \vspace{0.5ex}

\hspace{.8\funcindent}\begin{boxedminipage}{\funcwidth}

    \raggedright \textbf{closeViewer}()

    \vspace{-1.5ex}

    \rule{\textwidth}{0.5\fboxrule}
\setlength{\parskip}{2ex}
    Close the viewer (not hide as show(false)

\setlength{\parskip}{1ex}
    \end{boxedminipage}

    \label{tracetool:TTrace:clearAll}
    \index{tracetool \textit{(module)}!tracetool.TTrace \textit{(class)}!tracetool.TTrace.clearAll \textit{(static method)}}

    \vspace{0.5ex}

\hspace{.8\funcindent}\begin{boxedminipage}{\funcwidth}

    \raggedright \textbf{clearAll}()

    \vspace{-1.5ex}

    \rule{\textwidth}{0.5\fboxrule}
\setlength{\parskip}{2ex}
    Clear all traces

\setlength{\parskip}{1ex}
    \end{boxedminipage}

    \label{tracetool:TTrace:stop}
    \index{tracetool \textit{(module)}!tracetool.TTrace \textit{(class)}!tracetool.TTrace.stop \textit{(static method)}}

    \vspace{0.5ex}

\hspace{.8\funcindent}\begin{boxedminipage}{\funcwidth}

    \raggedright \textbf{stop}()

    \vspace{-1.5ex}

    \rule{\textwidth}{0.5\fboxrule}
\setlength{\parskip}{2ex}
    Stop sub-system before leaving your program. You may call 
    ttrace.flush() before stop()

\setlength{\parskip}{1ex}
    \end{boxedminipage}

    \label{tracetool:TTrace:flush}
    \index{tracetool \textit{(module)}!tracetool.TTrace \textit{(class)}!tracetool.TTrace.flush \textit{(static method)}}

    \vspace{0.5ex}

\hspace{.8\funcindent}\begin{boxedminipage}{\funcwidth}

    \raggedright \textbf{flush}()

    \vspace{-1.5ex}

    \rule{\textwidth}{0.5\fboxrule}
\setlength{\parskip}{2ex}
    flush remaining traces to the viewer

\setlength{\parskip}{1ex}
    \end{boxedminipage}


\large{\textbf{\textit{Inherited from object}}}

\begin{quote}
\_\_delattr\_\_(), \_\_format\_\_(), \_\_getattribute\_\_(), \_\_hash\_\_(), \_\_init\_\_(), \_\_new\_\_(), \_\_reduce\_\_(), \_\_reduce\_ex\_\_(), \_\_repr\_\_(), \_\_setattr\_\_(), \_\_sizeof\_\_(), \_\_str\_\_(), \_\_subclasshook\_\_()
\end{quote}

%%%%%%%%%%%%%%%%%%%%%%%%%%%%%%%%%%%%%%%%%%%%%%%%%%%%%%%%%%%%%%%%%%%%%%%%%%%
%%                              Properties                               %%
%%%%%%%%%%%%%%%%%%%%%%%%%%%%%%%%%%%%%%%%%%%%%%%%%%%%%%%%%%%%%%%%%%%%%%%%%%%

  \subsubsection{Properties}

    \vspace{-1cm}
\hspace{\varindent}\begin{longtable}{|p{\varnamewidth}|p{\vardescrwidth}|l}
\cline{1-2}
\cline{1-2} \centering \textbf{Name} & \centering \textbf{Description}& \\
\cline{1-2}
\endhead\cline{1-2}\multicolumn{3}{r}{\small\textit{continued on next page}}\\\endfoot\cline{1-2}
\endlastfoot\multicolumn{2}{|l|}{\textit{Inherited from object}}\\
\multicolumn{2}{|p{\varwidth}|}{\raggedright \_\_class\_\_}\\
\cline{1-2}
\end{longtable}


%%%%%%%%%%%%%%%%%%%%%%%%%%%%%%%%%%%%%%%%%%%%%%%%%%%%%%%%%%%%%%%%%%%%%%%%%%%
%%                            Class Variables                            %%
%%%%%%%%%%%%%%%%%%%%%%%%%%%%%%%%%%%%%%%%%%%%%%%%%%%%%%%%%%%%%%%%%%%%%%%%%%%

  \subsubsection{Class Variables}

    \vspace{-1cm}
\hspace{\varindent}\begin{longtable}{|p{\varnamewidth}|p{\vardescrwidth}|l}
\cline{1-2}
\cline{1-2} \centering \textbf{Name} & \centering \textbf{Description}& \\
\cline{1-2}
\endhead\cline{1-2}\multicolumn{3}{r}{\small\textit{continued on next page}}\\\endfoot\cline{1-2}
\endlastfoot\raggedright n\-a\-t\-i\-v\-e\-C\-l\-a\-s\-s\-e\-s\- & \raggedright \textbf{Value:} 
{\tt \texttt{[}\texttt{]}}&\\
\cline{1-2}
\raggedright R\-g\-b\-C\-o\-l\-o\-r\-s\- & \raggedright \textbf{Value:} 
{\tt \texttt{\{}\texttt{'}\texttt{Aqua}\texttt{'}\texttt{: }\texttt{(}0\texttt{, }255\texttt{, }255\texttt{)}\texttt{, }\texttt{'}\texttt{Black}\texttt{'}\texttt{: }\texttt{(}0\texttt{, }0\texttt{, }0\texttt{)}\texttt{, }\texttt{'}\texttt{Blue}\texttt{'}\texttt{: }\texttt{(}0\texttt{, }0\texttt{...}}&\\
\cline{1-2}
\raggedright T\-r\-a\-c\-e\-C\-o\-n\-s\-t\- & \raggedright \textbf{Value:} 
{\tt \texttt{\{}\texttt{'}\texttt{CST\_ACTION\_CLEAR\_ALL}\texttt{'}\texttt{: }10\texttt{, }\texttt{'}\texttt{CST\_ACTION\_CLOSE\_WIN}\texttt{'}\texttt{: }11\texttt{, }\texttt{...}}&\\
\cline{1-2}
\raggedright o\-p\-t\-i\-o\-n\-s\- & \raggedright \textbf{Value:} 
{\tt TTraceOptions()}&\\
\cline{1-2}
\raggedright i\-n\-t\-e\-r\-n\-a\-l\-s\- & \raggedright \textbf{Value:} 
{\tt Internals()}&\\
\cline{1-2}
\raggedright w\-i\-n\-T\-r\-a\-c\-e\- & \raggedright \textbf{Value:} 
{\tt WinTrace("\_")}&\\
\cline{1-2}
\raggedright w\-a\-t\-c\-h\-e\-s\- & \raggedright \textbf{Value:} 
{\tt WinWatch()}&\\
\cline{1-2}
\raggedright d\-e\-b\-u\-g\- & \raggedright \textbf{Value:} 
{\tt TTrace.winTrace.debug}&\\
\cline{1-2}
\raggedright w\-a\-r\-n\-i\-n\-g\- & \raggedright \textbf{Value:} 
{\tt TTrace.winTrace.warning}&\\
\cline{1-2}
\raggedright e\-r\-r\-o\-r\- & \raggedright \textbf{Value:} 
{\tt TTrace.winTrace.error}&\\
\cline{1-2}
\end{longtable}

    \index{tracetool \textit{(module)}!tracetool.TTrace \textit{(class)}|)}
    \index{tracetool \textit{(module)}|)}
